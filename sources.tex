\begin{thebibliography}{00}
\addcontentsline{toc}{chapter}{Список использованных источников}

% https://www.overleaf.com/learn/latex/Bibliography_management_with_bibtex
% Use: \cite{label00}

\bibitem{label22} Аксаков А.Г., Свистунов А.Н., Бабич И.Н., Алтухов С.В., и др. Законопроект № 341257-8 О внесении изменений в отдельные законодательные акты Российской Федерации (в части установления экспериментальных правовых режимов в сфере цифровых инноваций на финансовом рынке) [Электронный ресурс], Система обеспечения законодательной деятельности. -- URL: https://sozd.duma.gov.ru/bill/341257-8 (дата обращения: 20.04.2024).

\bibitem{label29} Албычев А.С., Ильин Д.Ю. ВЫБОР СТЕКА ТЕХНОЛОГИЙ ВЫЧИСЛИТЕЛЬНОЙ ИНФРАСТРУКТУРЫ ДЛЯ ЭКСПЕРИМЕНТАЛЬНЫХ ИССЛЕДОВАНИЙ ЦИФРОВЫХ ВАЛЮТ // International Journal of Open Information Technologies. 2023. №4. URL: https://cyberleninka.ru/article/n/vybor-steka-tehnologiy-vychislitelnoy-infrastruktury-dlya-eksperimentalnyh-issledovaniy-tsifrovyh-valyut (дата обращения: 10.06.2024).

\bibitem{label23} Биржа Mt.Gox: крупнейший взлом в истории криптовалют [Электронный ресурс], ForkLog журнал о биткоине, технологии блокчейн и цифровой экономике. -- URL: https://forklog.com/cryptorium/birzha-mt-gox-krupnejshij-vzlom-v-istorii-kriptovalyut (дата обращения: 10.06.2024).

\bibitem{label15} Блокчейн в корпоративной архитектуре: дань моде или необходимость? [Электронный ресурс], Хабр. -- URL:  https://habr.com/ru/companies/web3\_tech/articles/658447/ (дата обращения: 01.04.2024).

\bibitem{label19} Болдачев А.В. Миф о консенсусе [Электронный ресурс], Издательство Открытые системы. -- URL:  https://www.osp.ru/os/2019/02/13054960 (дата обращения: 20.04.2024).

\bibitem{label25} Британская «дочка» Binance отказалась от регистрации в FCA [Электронный ресурс], ForkLog журнал о биткоине, технологии блокчейн и цифровой экономике. -- URL: https://forklog.com/news/britanskaya-dochka-binance-otkazalas-ot-registratsii-v-fca (дата обращения: 10.06.2024).

\bibitem{label30} Булыга Р.П., Сафонова И.В. Ерансформация методологии аудита в связи с использованием технологий блокчейн и DLT // Учет. Анализ. Аудит. 2021. №5. URL: https://cyberleninka.ru/article/n/transformatsiya-metodologii-audita-v-svyazi-s-ispolzovaniem-tehnologiy-blokcheyn-i-dlt (дата обращения: 10.06.2024).

\bibitem{label26} В Греции задержан россиянин, обвиняемый в отмывании \$4 млрд в биткоинах [Электронный ресурс], ForkLog журнал о биткоине, технологии блокчейн и цифровой экономике. -- URL: https://forklog.com/news/v-gretsii-zaderzhan-rossiyanin-obvinyaemyj-v-otmyvanii-4-mlrd-v-bitkoinah (дата обращения: 10.06.2024).

\bibitem{label1} Голосование по поправкам в Конституцию Российской Федерации [Электронный ресурс], Официальный сайт мэра Москвы. -- URL: https://www.mos.ru/city/projects/vote2020/\#rec168487887\#!/tab/207311641-4 (дата обращения: 01.04.2024).

\bibitem{label10} Егорова М.А., Белых В.С., Решетникова С.Б. Технология блокчейн: перспективы применения и значение для целей развития информационного общества // Юрист. 2019. N 7. С. 4-9.

\bibitem{label38} Использование технологии DLT на финансовых рынках поможет сэкономить \$100 млрд [Электронный ресурс], crypto.ru. -- URL: https://crypto.ru/ispolzovanie-dlt-pomozhet-sekonomit-100-mlrd/ (дата обращения: 20.04.2024).

\bibitem{label33} Моисеев А. Совершенствуем процедуру Know Your Customer с помощью блокчейна [Электронный ресурс], Лаборатория Касперского. -- URL: https://www.kaspersky.ru/blog/kyc-blockchain/27528/ (дата обращения: 20.04.2024).

\bibitem{label9} Нагродская В.Б. Новые технологии (блокчейн / искусственный интеллект) на службе права: научно-методическое пособие / под ред. Л.А. Новоселовой. М.: Проспект, 2019. 128 с.

\bibitem{label16} Программа льготного кредитования субъектов малого и среднего предпринимательства в 2019 – 2024 годах [Электронный ресурс], Государственная информационная система промышленности. -- URL: https://gisp.gov.ru/nmp/measure/9604994 (дата обращения: 01.04.2024).

\bibitem{label35} Полешкина И.О., Васильева Н.В. Технология Blockchain как инструмент управления цепями поставок с участием воздушного транспорта // Научный вестник МГТУ ГА. 2020. №2. URL: https://cyberleninka.ru/article/n/tehnologiya-blockchain-kak-instrument-upravleniya-tsepyami-postavok-s-uchastiem-vozdushnogo-transporta (дата обращения: 20.04.2024).

\bibitem{label24} Регулятор: QuadrigaCX обанкротилась из-за мошенничества ее основателя [Электронный ресурс], ForkLog журнал о биткоине, технологии блокчейн и цифровой экономике. -- URL: https://forklog.com/news/regulyator-quadrigacx-obankrotilas-iz-za-moshennichestva-ee-osnovatelya (дата обращения: 10.06.2024).

\bibitem{label32} Смарт-контракты: их роль и работа в блокчейне [Электронный ресурс], plisio.net криптовалютный платежный шлюз. -- URL: https://plisio.net/ru/blog/smart-contracts-their-role-and-operation-in-blockchain (дата обращения: 20.04.2024).

\bibitem{label34} Сятчихин А.В. Смарт-контракт: возможности и условия реализации технологии на примере продажи недвижимости // Пермский юридический альманах. 2019. №2. URL: https://cyberleninka.ru/article/n/smart-kontrakt-vozmozhnosti-i-usloviya-realizatsii-tehnologii-na-primere-prodazhi-nedvizhimosti (дата обращения: 20.04.2024).

\bibitem{label37} Орлов О.В. Экономический механизм и институты международных расчётов в криптовалюте // Финансовые рынки и банки. 2022. №8. URL: https://cyberleninka.ru/article/n/ekonomicheskiy-mehanizm-i-instituty-mezhdunarodnyh-raschyotov-v-kriptovalyute (дата обращения: 20.04.2024).

\bibitem{label17} Цифровая Платформа Распределенного Реестра ФНС России (ЦПРР ФНС России). Подсистема администрирования. Техническое описание [Электронный ресурс], Федеральная Налоговая Служба. -- URL: https://data.nalog.ru/html/sites/www.new.nalog.ru/docs/uis/technical\_specification.docx  (дата обращения: 01.04.2024).

\bibitem{label27} Что такое атомарные свопы? [Электронный ресурс], ForkLog журнал о биткоине, технологии блокчейн и цифровой экономике. -- URL: https://forklog.com/cryptorium/chto-takoe-atomarnye-svopy (дата обращения: 10.06.2024).

\bibitem{label36} Что такое MakerDAO (MKR) и стейблкоин DAI? [Электронный ресурс], ForkLog журнал о биткоине, технологии блокчейн и цифровой экономике. -- URL: https://forklog.com/cryptorium/chto-takoe-makerdao (дата обращения: 20.04.2024).

\bibitem{label18} Что такое Proof-of-Work и Proof-of-Stake? [Электронный ресурс], ForkLog журнал о биткоине, технологии блокчейн и цифровой экономике. -- URL: https://forklog.com/chto-takoe-proof-of-work-i-proof-of-stake (дата обращения: 20.04.2024).

\bibitem{label39} Шилов К.Д., Зубарев А.В. Блокчейн и распределенные реестры как виды баз данных // Инновации. 2018. №12 (242). URL: https://cyberleninka.ru/article/n/blokcheyn-i-raspredelennye-reestry-kak-vidy-baz-dannyh (дата обращения: 20.04.2024).

\bibitem{label60} A Deep Dive Into Blockchain Scalability [Электронный ресурс], crypto.com. -- URL: https://crypto.com/university/blockchain-scalability (дата обращения: 20.04.2024).

\bibitem{label28} BitShares Core Release 3.0.0 [Электронный ресурс], Открытый реестр кода BitShares. -- URL: https://github.com/bitshares/bitshares-core/releases/tag/3.0.0 (дата обращения: 20.04.2024).

\bibitem{label3} Blockchain and distributed ledger technologies Vocabulary [Электронный ресурс], Официальный сайт Международной организации по стандартизации ISO. -- URL: https://www.iso.org/obp/ui/\#iso:std:iso:22739:ed-1:v1:en (дата обращения: 01.04.2024).

\bibitem{label11} Buterin Vitalik. An Incomplete Guide to Rollups [Электронный ресурс], Vitalik Buterin's website. -- URL:  https://vitalik.ca/general/2021/01/05/rollup.html (дата обращения: 20.04.2024).

\bibitem{label31} Gavin Wood (Dr.). Ethereum: A Secure Decentralised Generalised Transaction Ledger [Электронный ресурс], Открытый репозиторий проекта Ethereum. -- URL  https://ethereum.github.io/yellowpaper/paper.pdf (дата обращения: 20.05.2024).

\bibitem{label61} Graphene::App [Электронный ресурс], BitShares Developers Portal. -- URL: https://dev.bitshares.works/en/master/api/namespaces/app.html (дата обращения: 01.06.2024).

\bibitem{label21} Introduction [Электронный ресурс], TON. -- URL: https://docs.ton.org/develop/smart-contracts\#programming-languages (дата обращения: 10.06.2024).

\bibitem{label12} ISO/IEC 22123-2:2023 Information technology -- Cloud computing. Part 2: Concepts [Электронный ресурс], ISO: the International Organization for Standardization. -- URL: https://www.iso.org/standard/80351.html (дата обращения: 20.04.2024).

\bibitem{label14} Joseph Poon, Thaddeus Dryja. The Bitcoin Lightning Network: Scalable Off-Chain Instant Payments [Электронный ресурс], Lightning Network. -- URL: https://lightning.network/lightning-network-paper.pdf (дата обращения: 01.04.2024).

\bibitem{label2} Nakamoto Satoshi. Bitcoin: A Peer-to-Peer Electronic Cash System [Электронный ресурс], The U.S. Sentencing Commission --  URL: https://www.ussc.gov/sites/default/files/pdf/training/annual-national-training-seminar/2018/Emerging\_Tech\_Bitcoin\_Crypto.pdf (дата обращения: 01.04.2024).

\bibitem{label8} Primavera De Filippi and Aaron W right. Blockchain and the law: the rule of code. Cambridge, Massachusetts: Harvard University Press, 2018. 312 p.

\bibitem{label20} Recurring \& Scheduled Payments [Электронный ресурс], BitShares. -- URL: https://bitshares.org/recurring-scheduled-payments/ (дата обращения: 20.04.2024).

\end{thebibliography}