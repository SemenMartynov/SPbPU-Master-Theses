\chapter{Основные понятия и возможности распределённых реестров}

В разделе приведены основные понятия распределённых реестров (и технологии блокчейн, как честного случая), рассмотрены случаи применения распределённых решений в различных отраслях, а так же проведён краткий анализ возможностей и ограничений технологии. Особое внимание уделено проблеме интероперабельности сетей.

\section{Распределенные реестров и их влияние на различные отрасли}

Существующие на данный момент централизованные бизнес-реестры данных демонстрируют ряд недостатков, препятствующих эффективному функционированию современных бизнес-процессов. Низкая эффективность существующих реестров обусловлена их сложной структурой, что приводит к повышенным затратам на их поддержку и обслуживание. Высокая стоимость создания и поддержания реестров делает их недоступными для многих участников бизнес-процессов. Отсутствие прозрачности в процессе управления реестрами создает риски неправильного использования данных, в т.ч. случаи незаконного получения доступа к данным (утечки), что подрывает доверие к реестрам и увеличивает риски для бизнеса. Разрозненность реестров в системах разных участников бизнес-процессов приводит к несогласованности данных, что усложняет принятие информированных решений. Ошибки в данных из-за отсутствия синхронизации могут привести к неверным решениям и ущербу для бизнеса. Споры, возникающие из-за недостоверности данных, требуют дорогостоящих процедур урегулирования с привлечением аудиторов, что увеличивает затраты для участников бизнес-процессов\cite{label30}.

Технологии распределенных реестров (Distributed Ledger Technologies, DLT) представляют собой инновационную парадигму децентрализованного хранения и управления информацией. DLT системы функционируют на основе распределенной сети узлов, каждый из которых обладает идентичной копией реестра, исключая централизованный контроль\cite{label39}.

Когда мы говорим о любой распределённой системе, мы сталкиваемся с классическим набором ограничений -- узлы сети могут испытывать задержки в передаче данных, периодически становиться недоступными и даже проявлять злонамеренное поведение, посылая некорректные пакеты данных. Несмотря на эти потенциальные проблемы, распределённый реестр обязан обеспечивать консистентность сети, при условии, что большинство узлов функционируют корректно и имеют согласованную точку зрения на журнал событий\cite{label15}.

Для иллюстрации рассмотрим взаимодействие между крупным металлургическим заводом и нефтегазовой компанией. Завод производит трубы по заказу компании, качество которых фиксируется в специализированной системе. Далее трубы транспортируются по железной дороге, и оператор перевозок регистрирует их в своей системе. По прибытии трубы снова проходят проверку, результаты которой заносятся в соответствующую систему. На складе их статус также фиксируется. В итоге, в процессе участвует множество промежуточных систем (таких как Контур, 1C), что делает автоматизацию этого взаимодействия достаточно обоснованным.

Один из возможных подходов для решения этой задачи -- использование стандартных интеграций вида «точка-точка» (peer-to-peer) с каждым контрагентом. В случае с одним заводом и одной нефтегазовой компанией требуется одна интеграция. С добавлением транспортной компании количество интеграций увеличивается до трёх. По мере роста числа участников количество интеграций возрастает по формуле $ \frac{n(n-1)}{2} $, где $ n $ -- число участников. Ситуацию осложняет и тот факт, что многие системы на рынке не предусматривают единый формат передачи полных данных. Полные данные можно получить через API-интерфейс, а это подразумевает написание различных решений по интеграции. Крупные компании часто тратят годы и значительные ресурсы на такие проекты. Но с развитием бизнеса, подход "точка-точка" становится всё более громоздким, и покрыть все возможные случаи кодом становится сложнее, что приводит к необходимости решать проблемы через электронную почту, телефон и прочие средства.

Альтернативным подходом состоит в использовании единой базы данных в облаке (удалённом хранилище). В этом сценарии завод, нефтегазовая компания и другие контрагенты могли бы добавлять данные о трубах и изменять их статусы. Этот подход решает проблему согласования данных, но создаёт новые вопросы: кто будет отвечать за данные? Как быть с коммерческой тайной? Можно ли доверять облачному провайдеру? Участники готовы нести ответственность за свои данные, но редко доверяют их другим, особенно крупные компании, предпочитающие закрытые системы (не говоря о компаниях, которые должны охранять свои данные по закону).

Третий подходов к решению этой задачи это использование отраслевого решения от специализированной компании-разработчика, утверждающей, что она может решить все задачи, связанные с бизнес-процессами. В этом случае все взаимодействие будет происходить через одного поставщика решений. Большинство интеграций, включающих множество участников, реализуются именно через таких сервис-провайдеров. Однако возникает проблема выбора подходящего поставщика услуг. Например, при обмене кредитными историями подходящим провайдером мог бы выступить Сбербанк, что дало бы ему конкурентное преимущество. Чтобы избежать такой ситуации, необходимо привлечь независимого участника с особыми полномочиями, такого как бюро кредитных историй (БКИ), которое специализируется на обмене данными и разработке соответствующего программного обеспечения.

Тем не менее, подход с использованием сервис-провайдера имеет свои проблемы. Создание нового глобального игрока требует существенных затрат и в некоторых случаях изменений в законодательстве. Централизация данных в одном месте делает систему более уязвимой, поскольку сервис-провайдер может случайно или намеренно разгласить данные.

В итоге мы приходим к решению, основанному на распределенном реестре. Эта технология позволяет всем участникам процесса эффективно обмениваться данными, сохраняя при этом контроль над доступом к информации, а количество интеграций всегда будет соответствовать количеству участников. Использование DLT  потребует от компаний решения ряда проблем: потребуется обновить некоторые административные процессы и подготовить необходимую инфраструктуры внутри своей организации. Это явным образом означает, что каждая компания отвечает только за свою внутреннюю инфраструктуру. Надёжность сети сохраняется, даже если один из узлов выйдет из строя.

В последние годы наблюдается стремительный рост интереса к DLT, обусловленный рядом объективных преимуществ:
\begin{itemize}
	\item Повышенная безопасность. Децентрализованная архитектура DLT исключает единую точку отказа, что делает систему устойчивой к кибератакам, мошенничеству и цензуре.

	\item Прозрачность и доверие. Все участники сети имеют доступ к идентичной информации, что способствует повышению прозрачности и укреплению доверия между сторонами.

	\item Автоматизация и эффективность. DLT предоставляют возможности для автоматизации сложных процессов, что приводит к сокращению времени обработки данных и снижению операционных затрат.

	\item Новые возможности. DLT открывают новые горизонты для разработки инновационных решений, подобных смарт-контрактам, в различных секторах экономики.
\end{itemize}

DLT оказывают значительное влияние на трансформацию различных отраслей:
\begin{itemize}
	\item Финансовый сектор: DLT применяются для осуществления платежей, выпуска цифровых ценных бумаг, управления активами и предотвращения финансовых преступлений. Согласно исследованию TradFi, финансовые рынки на основе DLT, дадут возможность сэкономить до \$100 млрд издержек в год. компания Euroclear, базирующаяся в Брюсселе (Бельгия), запланировала запуск платформы для торговли облигациями на базе DLT\cite{label38}.

	\item Логистика и управление цепочками поставок: DLT обеспечивают прозрачность и отслеживаемость движения товаров на всех этапах цепочки поставок, оптимизируя логистические операции.

	\item Здравоохранение: DLT позволяют безопасно хранить и обмениваться конфиденциальными медицинскими данными, способствуя повышению качества медицинского обслуживания и диагностики.

	\item Образование: DLT могут использоваться для создания защищенной системы хранения и проверки сертификатов, дипломов и другой академической информации.

	\item Государственное управление: DLT применяются для хранения и верификации официальных документов, организации электронного голосования, борьбы с коррупцией и повышения эффективности государственных услуг. Блокчейн-технология открывает новые возможности для расширения механизмов реализации гражданских прав на участие в управлении публичными делами, включая избирательные права. Примером этого является проведение онлайн-голосования по поправкам в Конституцию Российской Федерации, которое прошло в Москве и Нижегородской области в июле 2020 года с использованием блокчейн-технологий\cite{label1}.
\end{itemize}

Интересным примером использования распределённых реестров является опыт проекта "Цифровая платформа распределенного реестра ФНС". В 2019 году правительство приняло программу льготного кредитования субъектов малого и среднего предпринимательства, позволяющую компаниям получать кредиты на льготных условиях, субсидируемые государством\cite{label16}. В программе содержалось ограничение, по которому одна компания могла иметь только одну активную заявку на кредит в каждом банке. Практическая реализация этого постановления выявила уязвимости, способствующие мошенничеству. Из-за задержек в обновлении данных в бюро кредитных историй (БКИ) компании могли подавать многочисленные заявки на кредиты в различных банках одновременно. Дополнительно, соблюдение банковской тайны и обмен конфиденциальной информацией потребовали значительных доработок существующих систем.

В ответ на эти вызовы было предложено использовать технологию блокчейн. Благодаря смарт-контрактам, уже на этапе подачи заявки на кредит можно внедрить ключевой инвариант: «один уникальный ИНН не может иметь две активные заявки». Данное решение предоставляет широкие функциональные возможности, позволяя создавать различные приложения на базе этой инфраструктуры без необходимости разработки новых интеграций. Например, был запущен проект единой базы машиночитаемых доверенностей\cite{label15}.

Анализируя техническую спецификацию этого проекта\cite{label17}, можно отметить, что в рамках этой платформы каждая заявка имеет открытую и закрытую часть. Открытая часть обрабатывается смарт-контрактами и включает статусы, идентификаторы, номера заявок, ИНН и ОГРН. Эта не конфиденциальная информация хранится в открытом виде в блокчейне. Закрытая часть включает конфиденциальные данные, такие как сумма кредита, количество сотрудников и кредитный лимит. В данном решении отсутствует необходимости работать с другими реестрами, но для бизнес-проектов часто требуется требуется взаимодействие с внешними данными, со сторонними блокчейнами.

Несмотря на то, что термины "распределенный реестр" (DLT) и "блокчейн" (Blockchain) часто используются взаимозаменяемо, важно провести четкое разграничение между этими понятиями. Blockchain представляет собой специфический тип DLT, где данные организованы в виде последовательной цепочки блоков.

Выделим основные различия:
\begin{itemize}[]
	\item Структура данных.\\
	Blockchain использует линейную структуру блоков, а DLT могут иметь различную архитектуру (направленный ациклический граф, распределенный реестр без блоков и т.д.).

	\item Неизменность данных.\\
	Все действия, осуществляемые в блокчейне, записывается в новый блок, который присоединяется к существующей цепочке. Удаление или изменение уже записанной транзакции невозможно. Это делает блокчейн высоко достоверным и прозрачным, поскольку вся информация о прошедших транзакциях доступна для всех участников сети в неизменном виде.

	\item Механизмы консенсуса.\\
	Blockchain обычно используют консенсусные механизмы, такие как Proof-of-Work (PoW) или Proof-of-Stake (PoS), а DLT могут применять иные подходы к достижению консенсуса.

	\item Область применения.\\
	Blockchain чаще всего ассоциируется с токенезированными активами (в частности, криптовалютами), а DLT имеют более широкий спектр применения, охватывающий различные отрасли.
\end{itemize}

Отсюда можно заключить, что DLT -- это общий термин, обозначающий любую технологию распределенного реестра, а blockchain представляет собой конкретную реализацию DLT, использующую цепочку блоков для хранения данных. Узлы сети блокчейн взаимодействуют друг с другом и используют протокол консенсуса для согласования содержимого реестра. Этот процесс обеспечивает точную репликацию реестра у всех участников сети, снижая риски мошенничества, поскольку для изменения данных требуется одновременная модификация у многих узлов.

\section{Возможности блокчейн}

Стандарт ISO 22739:2020 «Технологии блокчейн и распределенного реестра. Словарь» определяет блокчейн как распределенный реестр организованный в последовательную цепочку путём добавления подтвержденных блоков с использованием криптографических ссылок\cite{label3}, ничего не говоря об области применения технологии. Однако автор блокчейна и протокола биткоин ещё в 2008 году явным образом указывал, что Bitcoin это замена традиционным деньгам \cite{label2}. Такой высокий уровень ожиданий от технологии может быть только при беспрецедентном уровне безопасности и надёжности.

Безопасность данных в блокчейне обеспечивается за счет децентрализованного хранения информации, что представляет собой ключевое отличие от традиционных подходов к хранению данных. В традиционных системах хранения данных, таких как централизованные базы данных, информация обычно хранится на серверах одной организации или компании. Это создает центральную точку уязвимости, где любое нарушение безопасности или несанкционированный доступ к серверам способно привести к компрометации данных. Кроме того, централизованные системы подвержены риску отказа или выхода из строя, что может привести к потере или повреждению данных\cite{label8}.

В блокчейне данные хранятся на множестве независимых компьютеров, образуя распределенную сеть. Каждый узел в сети содержит полную копию всех данных, что обеспечивает их дублирование и распределение по всей сети. Это делает блокчейн устойчивым к единичным отказам или атакам на отдельные узлы, так как данные сохраняются и остаются доступными на других узлах. Даже если один или несколько узлов выйдут из строя (либо подвергнутся атаке), информация все равно будет доступна из оставшихся узлов.

Кроме того, децентрализованное хранение данных в блокчейне делает его менее подверженным целенаправленным атакам или взломам, поскольку для компрометации системы необходимо взломать большое количество узлов, а это является сложной и затратной задачей. Это позволяет блокчейну обеспечивать более высокий уровень безопасности данных в сравнении с традиционными централизованными системами хранения.

Если один из узлов в блокчейне попытается изменить данные, механизм консенсуса вступит в действие для проверки и подтверждения правильности информации\cite{label9}.

\subsection{Механизм консенсуса}

Механизм консенсуса -- это процесс, с помощью которого участники в блокчейне достигают согласия относительно того, какие транзакции действительны и какие должны быть добавлены в блокчейн. В зависимости от конкретной реализации блокчейна, механизм консенсуса различаются, но общая цель остается одной - обеспечить единство и согласованность данных в сети.

В профессиональной среде разработчиков и аналитиков, алгоритмами консенсуса часто называют Proof-of-Work (PoW, доказательство работы) и Proof-of-Stake (PoS, доказательство владения долей)\cite{label18}. Первоначально концепция PoW была предложена для защиты от почтового спама. Принцип заключался в том, что отправитель должен потратить несколько секунд процессорного времени на выполнение определенной задачи, такой как поиск хеша заголовка сообщения с несколькими нулями в начале. Проверка правильности хеша выполнялась мгновенно получателем. Это делало массовую рассылку писем нерентабельной для спамеров, так как требовало бы значительных вычислительных ресурсов.

Механизм PoW позволил ответить на вопрос, какой блок будет следующим в цепи блокчейна: это определялось тем, кто первым найдет требуемый хеш. Механизм PoS так же отвечает на вопрос о том, чья очередь генерировать новый блок, но здесь приоритет отдается владельцам наибольшего количества токенов. Но сами по себе эти механизмы не обеспечивают консенсуса в распределенной среде, так как существует вероятность одновременного появления нескольких блоков с правильными хешами/подписями.

Эти параллельные цепочки могут содержать невалидные транзакции, создаваемые злоумышленниками для проведения двойных трат. В итоге возникают альтернативные цепочки блокчейна. В такой ситуации узлы сети голосуют, поддерживая одну из цепочек. Побеждает цепочка, которая будет наиболее длинной и имеет наибольшую поддержку, измеряемую количеством узлов, вычислительной мощностью или долей токенов. Узлы, голосовавшие за другую цепочку, вынуждены отступить и принять блоки победившей цепи. Именно это «мнение большинства» и было указано автором биткоина как способ достижения консенсуса\cite{label2}.

Консенсус, согласно Накамото обеспечивается "голосование длиной цепочки форка", а PoW и PoS не являются непосредственно алгоритмами консенсуса, но играют важную роль в его реализации, способствуя установлению единого мнения о текущем состоянии блокчейна и обеспечивая стабильность и безопасность сети\cite{label19}. Будем говорить о PoW и PoS как о механизмах обеспечения консенсуса.

Механизм консесуса Proof of Work (PoW) применяется в блокчейне Bitcoin. В этой системе узлы, называемые майнерами, решают сложные математические задачи, чтобы добавить новый блок транзакций в цепочку. Если один из узлов попытается изменить данные в блоке, это потребует перерасчета хэша для этого блока и всех последующих блоков, что крайне затруднительно из-за сложности задач и необходимости контроля над большинством вычислительной мощности в сети. Поэтому майнеры будут отклонять измененные данные, и изменения не будут приняты.

В блокчейнах, использующем PoW, производители блоков конкурируют за право добавления нового блока транзакций в цепочку. Для этого им приходится решать сложные криптографические задачи, что требует значительных вычислительных ресурсов и, следовательно, энергии. Процесс майнинга в PoW потребляет огромные объемы электроэнергии, поскольку требует мощных компьютерных систем, работающих непрерывно для решения задач и создания новых блоков. Поскольку большинство энергии производится из невозобновляемых источников, высокое энергопотребление блокчейна может иметь негативное воздействие на окружающую среду и климат, увеличивая выбросы парниковых газов и прочие формы загрязнения.

Потребляемая энергия в масштабах всей планеты была так велика, что это подтолкнуло развитие механизма консенсуса Proof of Stake (PoS). Блокчейн Ethereum совершил переход к PoS уже после своего запуска, изначально он работал на PoW. Идея подхода Proof of Stake состоит в том, что право на управление блокчейном (производство следующего блока) имеют участники с наибольшей долей монет. Чтобы стать производителем блока, пользователь должен заблокировать свои, или привлеченные у других пользователей, средства. За производство блоков выплачивается вознаграждение. Но если производитель блоков пытается отойти от установленного протокола и подделать какие-то транзакции, то эти действия будут обнаружены другими участниками сети, а сам он потеряет те средства, которые отправил в залог. Сейчас этот подход используют ведущие криптовалютные платформы: Ethereum, Cardano, Solana и многие другие.

Механизм консенсуса в блокчейне обеспечивает надежность и целостность данных, предотвращая несанкционированные изменения и обеспечивая единство в сети.

\subsection{Безопасность}

Проблемы безопасности в блокчейне включают в себя ряд угроз и уязвимостей, которые могут быть использованы злоумышленниками для получения несанкционированного доступа к активам или данным.

Криптографические примитивы — неотъемлемый компонент безопасности блокчейна..

\begin{enumerate}
	\item Хеш-функции.\\Хеш-функция преобразует входные данные (блок информации) в уникальный цифровой отпечаток фиксированной длины, называемый хешем. Даже малейшее изменение входных данных приводит к совершенно другому хешу. Каждый блок в блокчейне содержит хеш предыдущего блока, образуя непрерывную цепочку. Это делает практически невозможным изменение данных в блоке, не изменив хеши всех последующих блоков, что потребовало бы огромных вычислительных ресурсов.

	\item Асимметричное шифрование.\\Используется пара ключей: закрытый и открытый. Закрытый ключ известен только владельцу, а открытый ключ доступен всем. Данные, зашифрованные открытым ключом, можно расшифровать только соответствующим закрытым ключом. Асимметричное шифрование используется при подпись транзакций: Закрытый ключ используется для создания цифровой подписи транзакции, а открытый ключ используется для проверки подписи транзакции и подтверждения, что она была создана владельцем соответствующего закрытого ключа.

	\item Эллиптическая кривая (Elliptic Curve Cryptography, ECC).\\Это криптографический метод, использующий алгоритмы на основе эллиптических кривых для генерации ключей и подписи данных. ECC обеспечивает ту же стойкость к взлому, что и RSA, но при меньших размерах ключей, что делает его более эффективным для блокчейна, где экономия места играет важную роль.

	\item Дерево Меркла (Merkle Tree).\\ Merkle Tree - это структура данных, которая позволяет эффективно проверять целостность большого объема данных. Хеши транзакций в блоке организованы в древовидную структуру, где каждый узел содержит хеш своих дочерних узлов. Корневой узел дерева содержит хеш всех транзакций в блоке. Merkle Tree позволяет быстро проверить, присутствует ли определенная транзакция в блоке, без необходимости сканирования всего блока.

	\item Доказательство работы (Proof of Work, PoW).\\ Хотя, как мы рассмотрели выше, PoW не является криптографическим алгоритмом в строгом смысле, он представляет собой криптографически обеспеченный механизмом консенсуса, используемый в блокчейне Bitcoin. PoW требует от майнеров выполнения вычислительно сложной задачи, что обеспечивает безопасность сети блокчейна и предотвращает манипуляции с данными.
\end{enumerate}

Эти криптографические примитивы играют важную роль в обеспечении безопасности и неизменности данных в блокчейне. Учитывая вычислительную сложность криптографических алгоритмов, правильнее сказать, что неизменность данных возникает как результат крайней дороговизны любой попытки сфальсифицировать данные\cite{label10}. Если блокчейн недостаточно децентрализован и злоумышленник контролирует более 50\% вычислительной мощности сети, он сможет провести так называемую "51\% атаку", что позволяет ему контролировать процесс верификации транзакций и проводить двойные расходы.

Если сам блокчейн строится на хорошо изученных и проверенных примитивах, а в некоторых случаях можно даже провести независимый аудит безопасности, то слабым звеном становится сам пользователь -- держать приватного ключа.

Приватные ключи используются для подписания транзакций в блокчейне и представляют собой основной механизм защиты активов пользователя. Похищение приватных ключей ведёт к потере доступа к активам либо их краже. Злоумышленники могут использовать различные методы, такие как фишинг или вредоносное ПО, для получения доступа к приватным ключам. Кроме того, злоумышленники могут использовать социальную инженерию для манипулирования пользователями блокчейна и получения доступа к их активам или данным. Например, они могут создавать фальшивые веб-сайты или представляться за доверенных лиц, чтобы обмануть пользователей и получить их приватные ключи. Интерес злоумышленников к приватным ключам растёт с повышением стоимости цифровых активов, которые управляются этим ключом.

\subsection{Экономическая модель}

Финансовая модель блокчейна основана на распределенной бухгалтерской книге, которая обеспечивает прозрачность, безопасность и децентрализацию транзакций. В этой модели участники сети могут обмениваться ценностями без необходимости в посредниках, таких как банки, что особенно важно в условиях нестабильности мировой системы\cite{label37}.

Основополагающими элементами финансовой модели блокчейна выступают:

\begin{enumerate}
	\item Создание цифровых активов. Блокчейн позволяет создавать цифровые активы, такие как криптовалюты и токены. Эти активы могут представлять собой как непосредственно валюту (коины), так и различные права и привилегии (токены).

	\item Распределение активов. Цифровые активы могут быть распределены через различные механизмы, такие как майнинг, стейкинг, а также первичное размещение монет (ICO) или токенов (ITO).

	\item Использование активов. Участники сети могут использовать цифровые активы для проведения транзакций, участия в управлении проектом, получения доступа к услугам или продуктам и других целей.
\end{enumerate}

Говоря про активы, мы имеем в виду как токены так и коины. Стоит отметить, что эти активы (за некоторым исключением) обеспечены какой-то пользой, которую они создают для общества. Это привело к интересу со стороны инвесторов, которые пожелали вложить свои деньги в эти активы, а сами активы стали обладать реальной стоимостью.

Стоит рассмотреть основные различия между токенами и коинами, которые являются фундаментальными элементами финансовой модели блокчейна.

Коины представляют собой цифровые валюты, которые функционируют на собственной блокчейн-платформе. Например, BTC работает на блокчейне Bitcoin, а ETH -- на блокчейне Ethereum. Коины часто используются в качестве средства обмена и накопления стоимости. Они могут служить аналогом традиционных валют в цифровом мире. Так же, коины часто создаются через процесс майнинга (выполнения работы по поддержанию сети, которая требует затрат вычислительных ресурсов) или стейкинга (блокировки определенного количества монет для получения вознаграждений).

Токены представляют собой цифровые активы, которые выпускаются на существующих блокчейн-платформах, таких как Ethereum, Binance Smart Chain или другие. Это позволяет разработчикам создавать токены без необходимости создавать новый блокчейн. Токены могут выполнять множество функций, включая право голоса (governance tokens), доступ к услугам или продуктам (utility tokens), и представление доли в проекте или активе (security tokens).

Существующие внутри блокчейнов коины функционируют как самостоятельные валюты, в то время как токены предлагают разнообразие функциональностей, которые могут быть заданы смарт-контрактом. При этом любые цифровые активы (как токены, так и коины) существую только в рамках своей экосистемы, т.е. технически невозможно перенести активы из одного блокчейна в другой. Можно только произвести обмен, с полагаясь на доверенную сторону, либо используя другие механизмы, такие как смарт-контакты.

\subsection{Смарт-контакты}

Идея смарт-контрактов была впервые предложена криптографом Ником Сабо (Nick Szabo) в 1994 году. Сабо, описал смарт-контракты как протоколы для выполнения условий контракта с использованием компьютерных программ. Его концепция предполагала, что смарт-контракты могли бы автоматически выполнять и контролировать выполнение условий договора, снижая необходимость в традиционных посредниках, таких как юристы и нотариусы. Для выполнения условий, контракт должен иметь доступ к ресурсам, которые ценны для участвующих сторон.

Концепция смарт-контрактов оставалась в значительной степени теоретической, пока не появился Ethereum - блокчейн-платформа с поддержкой смарт-контрактов, запущенная в 2015 году Виталиком Бутериным. Ethereum предоставил практический инструмент для создания и развертывания децентрализованных приложений (dApps), благодаря своему встроенному языку Solidity.

Блокчейн Ethereum предоставил виртуальную машину (Ethereum Virtual Machine, EVM) для исполнения смарт-контрактов и проверки результатов в распределенной среде. Выполнение условий смарт-контракта не требует вмешательства человека, а значит снижает риск человеческой ошибки и ускоряет выполнение сделок.

Смарт-контракты представляют собой программный код, который хранится в открытом, распределенном публичном реестре блокчейна. Любой желающий может просмотреть исходный код смарт-контракта и убедиться, что он реализует заявленную функциональность. Это позволяет всем участникам проверить контракт на наличие ошибок, уязвимостей или нежелательного поведения до его развертывания.

После того как смарт-контракт развернут в блокчейне, его код становится неизменным и его невозможно модифицировать без согласия всех участников сети. Это означает, что все стороны могут быть уверены, что логика контракта останется неизменной и будет выполняться строго так, как она была первоначально задумана.

Все транзакции и взаимодействия со смарт-контрактами записываются в публичный распределенный реестр блокчейна. Каждый может проверить полную историю вызовов контракта, переданные входные данные и результаты выполнения. Это обеспечивает беспрецедентный уровень аудита и подотчетности.

Смарт-контракты выполняются децентрализовано в сети блокчейна на множестве независимых узлов. Это гарантирует, что нет единой точки отказа или контроля, и что контракт будет исполняться в соответствии с заданной логикой без возможности внешнего вмешательства.

Благодаря этой прозрачности, участники могут самостоятельно убедиться в том, что смарт-контракт соответствует их ожиданиям и не содержит скрытых условий или изменений. Это повышает уровень доверия и снижает необходимость в посредниках, что является одним из основных преимуществ технологии блокчейн.

Безопасность смарт-контрактов строится на использовании криптографических примитивов, которые мы уже рассматривали ранее. Использование криптографических методов обеспечивает высокий уровень безопасности и защищенности данных.

Смарт-контракты обладают рядом уникальных возможностей, которые обеспечили их широкое применение в различных отраслях.

\begin{itemize}
	\item Смарт-контракты используются для автоматизации кредитования, страхования, управления активами и прочих финансовых операций. Например, платформа MakerDAO использует смарт-контракты для создания и управления стейблкоином DAI\cite{label36}.

	\item Смарт-контракты могут отслеживать движение товаров по цепочке поставок, автоматизировать оплату по мере достижения определенных этапов и обеспечивать подлинность продукции\cite{label35}.

	\item Смарт-контракты могут автоматизировать процессы купли-продажи недвижимости, аренды и управления имущественными правами, обеспечивая прозрачность и надежность сделок\cite{label34}.

	\item Смарт-контракты могут использоваться для управления цифровыми идентификациями, обеспечивая контроль доступа к данным и подтверждение личности\cite{label33}.

	\item  В играх смарт-контракты используются для управления виртуальными активами, проведения честных розыгрышей и создания игровых токенов\cite{label32}.
\end{itemize}

Говоря про смарт-контракты, стоит отметить, что они не обязательно должны выполняться в виртуальной машине. В качестве примера, можно назвать "встроенные смарт-контракты" блокчейна BitShares\cite{label20}. BitShares — это платформа для децентрализованных финансов (DeFi), которая использует смарт-контракты для автоматизации и обеспечения безопасности транзакций. В BitShares пользователи могут создавать и управлять цифровыми активами, торговать на децентрализованной бирже, а также выпускать и использовать BitAssets (SmartCoins), привязанные к стоимости реальных активов. Смарт-контракты BitShares обеспечивают автоматизацию этих процессов, гарантируя, что условия сделок выполняются точно и надежно. Это включает в себя создание пользовательских активов, управление аккаунтами и трансферы активов между пользователями с высоким уровнем безопасности и прозрачности. В сравнении с Ethereum, смарт-контракты BitShares, хотя и менее гибкие, более специализированы для финансовых операций и торговых платформ. Они встроены в ядро платформы и оптимизированы для высокой скорости и низких затрат на транзакции. Сами контракты написаны на языке C++ и компилируется как часть узла блокчейна, а пользователи могут взаимодействовать с ними передавая аргументы непосредственно во время работы блокчейна.

Несмотря на свои возможности, смарт-контракты имеют и определенные ограничения. Смарт-контракты часто зависят от внешних данных для выполнения условий. Эти данные предоставляются оракулами, которые сами могут быть уязвимы и подвержены ошибкам или манипуляциям. Публичные блокчейны, такие как Ethereum, могут столкнуться с проблемами масштабируемости при увеличении числа смарт-контрактов и транзакций, что приводит к увеличению комиссий и времени выполнения.

\subsection{Ограничения в распространении блокчейн технологии}

Проблема сложности использования в блокчейне относится к трудностям, с которыми сталкиваются обычные пользователи при работе с этой технологией из-за сложного интерфейса и недостатка знаний.

Изначально, блокчейн создавался и развивался в среде технических энтузиастов, которые не уделяли достаторное внимание вопросу пользовательского опыта и интерфейсов. Это привело к тому, что современные блокчейн-приложения и кошельки имеют сложные, технические-ориентированные интерфейсы, которые могут быть непонятными для обычных пользователей. Это отпугивает новичков и создаёт преграду для массового принятия блокчейна широкой аудиторией. Как минимум, пользователь блокчейн-кошельки должен иметь понимание концепции приватных и публичных ключей, адресов блокчейна (какой адрес к какой сети относится) и прочих технических аспектов, что препятствует широкому применению блокчейн технологии. Отсутствие достаточных знаний о принципах работы блокчейн ведёт к искаженному пониманию рисков и преимуществ использования блокчейна, а также к неудачным решениям при работе с криптовалютами и другими блокчейн-активами.

Необратимость транзакций в блокчейне -- одна из важных концепций, которую предстоит усвоить любому новому пользователю, которые представляет себе блокчейн как простую альтернативу банку. Она заключается в том, что после того, как транзакция была добавлена в блок и подтверждена сетью, ее невозможно отменить или изменить. Блокчейн функционирует на принципе децентрализации, что прямо означает отсутствие центрального органа или управляющего, который мог бы откатить или исправить проведенную транзакцию. Необратимость транзакций одна из причин прозрачности и надежности блокчейн сетей, так как исключает возможность манипуляций с данными после их записи в блок. Это особенно ценно для финансовых операций и других критически важных транзакций.

В связи с необратимостью транзакций принципиально важно, чтобы пользователи были внимательны и осуществляли проверку перед проведением транзакций. Это включает в себя двойную проверку адресов, обеспечение достаточных уровней безопасности приватных ключей и использование проверенных и надежных платформ и кошельков.

Отдельно стоит вопрос правового регулирования. В большинстве стран блокчейн и криптовалюты остаются относительно новыми явлениями, и законодательство в этой области не всегда достаточно развито или ясно определено. Долгое время блокчейн оставался в серой зоне, но по мере роста популярности технологии несовершенство законодательства и нормативных положений, которые регулируют использование и развитие блокчейн-технологий, становится всё более ощутимым. Правовые риски создают неопределенность для компаний и инвесторов, которые работают с блокчейн-технологиями, и затруднять развитие инноваций. Поскольку блокчейн-технологии могут использоваться для разнообразных целей, включая финансовые транзакции, управление цифровыми активами, учет данных и другое, имеет смысл разработать соответствующие нормативные рамки для защиты интересов потребителей и предотвращения мошенничества.

Использование криптовалют и блокчейн-активов вызывает вопросы в области налогообложения и финансового мониторинга. Необходимо разработать соответствующие законы и положения, которые позволят эффективно учитывать и контролировать операции с цифровыми активами.

С учетом того, что блокчейн обеспечивает прозрачность и неподдельность данных, необходимо балансировать эту прозрачность с защитой персональных данных и приватности пользователей. Регулирование должно обеспечить адекватный уровень защиты данных и соблюдение приватности.


Блокчейн -- это международная технология, и регулирование в этой области требует сотрудничества между различными странами и международными организациями. Отсутствие согласованных нормативных положений ведёт к появлению проблем для глобального принятия и использования блокчейна.Отсутствие четкого регулирования: Отсутствие четких правил и законодательства в сфере блокчейна создает неопределенность и риски для бизнеса и инвесторов.

\subsection{Масштабируемость}

С увеличением числа участников и объема транзакций возникают трудности с обработкой данных и поддержанием производительности сети. Блокчейны сталкиваются с серьезной проблемой масштабируемости, которая заключается в их способности обрабатывать большое количество транзакций в секунду (TPS) без ущерба для безопасности и децентрализации. Наиболее известные блокчейны, такие как Bitcoin и Ethereum, могут обрабатывать только ограниченное количество транзакций в секунду. По замерам исследователей, Bitcoin обрабатывает около 7 TPS, а Ethereum - около 30 TPS\cite{label60}. По мере увеличения популярности и использования этих сетей, число транзакций растет, что приводит к перегрузке сети, увеличению времени подтверждения транзакций и повышению комиссий за транзакции.

Для решения проблемы масштабирования, в Ethereum используется концепция газа для измерения стоимости выполнения операций или контрактов в сети. Газ необходим для оплаты вычислительных ресурсов, используемых для выполнения транзакций и смарт-контрактов. Он служит стимулом для майнеров, которые обрабатывают и подтверждают транзакции.

Каждая операция в Ethereum имеет фиксированное количество газа, необходимого для ее выполнения. Пользователи устанавливают лимит газа (максимальное количество газа, которое они готовы потратить на транзакцию) и цену газа (сумма, которую они готовы заплатить за единицу газа). Если сумма газа, необходимая для выполнения транзакции, превышает лимит газа, транзакция не будет выполнена. Цена газа варьируется в зависимости от спроса на вычислительные ресурсы сети\cite{label31}. Это можно назвать рыночным решением технической проблемы: когда возникает много желающих провести операцию, стоимость операции возрастает, и пользователь либо должен платить это повышенную цену, либо перенести свою операцию на менее загруженный период.

Ещё один подход к решению проблемы масштабирования это решения второго уровня (Layer 2 или L2). Они работают поверх основного блокчейна (Layer 1) и направлены на уменьшение нагрузки на основную сеть, улучшение пропускной способности и снижение комиссий за транзакции. Виталик Бутерин, автор Ethereum, выделяет три вида L2 решений: state channels, plasma, rollups\cite{label11}.

Для рекуррентных платежей удобно использовать \textbf{State Channels}. Он позволяют участникам проводить множество транзакций вне цепочки, выполняя расчёт взаимозачётов и фиксируя только начальное и конечное состояние на основном блокчейне. Примером данного подхода является Lightning Network для Bitcoin\cite{label14}. Ограничения State Channels состоят в том, что смарт-контракт в L1 не может взаимодействовать с внешним лицом, которое не было изначально заявлено как участник. Кроме того, в смарт-контракте требуется изначально заблокировать весь капитал, который будет участвовать в сделке, а если операция достаточно сложная, то заморозка капитала может быть тяжелым или невыполним условием.

\textbf{Plasma} использует дочерние цепочки (child chains), которые могут обрабатывать большое количество транзакций вне основного блокчейна, периодически отправляя результаты в основной блокчейн. Для этого у дочерней цепочки есть оператор (может быть как централизованный, так смарт-контрактом), который каждый интервал времени собирает все транзакции, полученные за установленный период, и высчитывает Merkle Tree, корень которого записывается на основной блокчейн. Множество транзакций сокращается до одного хэша. Это красивое техническое решение страдает от крайне высокой роли оператора. Его недобросовестные действия могут приводить к искажению результата.

Общий подход \textbf{rollups} напоминает тот, который был в plasma -- перемещение вычислений вне сети, сохраняя данные транзакций в основной сети. Но вместо оператора, используется соответствующий математический аппарат, который позволяет получить большую надёжность:
\begin{itemize}
	\item Optimistic Rollups.\\Идея состоит в том, чтобы собрать несколько транзакций в один пакет и отправить его в основной блокчейн. Пакет ожидает определенное время, чтобы кто-нибудь смог доказать, что транзакция недействительна. Если никто не предоставит доказательств, транзакции считаются действительными и фиксируется в основной блокчейн.

	\item ZK-Rollups (Zero-Knowledge Rollups).\\В этом подходе тоже выполняется вычисления вне основной сети, но используют доказательства с нулевым разглашением (Zero-Knowledge Proofs), чтобы гарантировать правильность вычислений. Это позволяет минимизировать размер данных, отправляемых в основной блокчейн.
\end{itemize}

Решения второго уровня призваны оптимизировать нагрузку на основную сеть и во всех этих решениях можно увидеть применения интероперабельности, но она применяется либо между различными кластерами одного блокчейна, либо между блокчейнами построенными на одном фреймворке, либо с использованием связки вне-блокчейн решений и программного интерфейса (API) блокчейна. Другими словами, ни один из этих подходов нельзя назвать устоявшимся решением для взаимодействия изолированных блокчейнов.

\section{Проблема интероперабельности блокчейн сетей}

С появлением биткоина и других криптовалют в информационно-технологическом сообществе возникла необходимость в обеспечении взаимодействия между различными платформами для обмена активами, находящимися в различных сетях распределенных реестров -- проблема интероперабельности. Сама интероперабельность определяется как способность двух или более компьютерных систем обмениваться информацией и взаимно использовать её\cite{label12}. В контексте блокчейн-технологий интероперабельность означает наличие протокола, который обеспечивает согласованность логических состояний в двух или более независимых распределенных реестрах. Это позволяет осуществлять обмен данными между различными сетями, что включает в себя не только криптовалюту, но и решение различных функциональных задач.

Отсутствие интероперабельности между блокчейн сетями приводит к ряду проблем, помимо уже рассмотренного вопроса масштабируемости.

\subsection{Форматы данных и смарт-контрактов}

Исторически сложилось, что различные блокчейн-проекты используют различные форматы данных и различные языки программирования для своих смарт-контрактов. Как результат, это ведёт к несовместимости смарт-контрактов между платформами и требуют дополнительных усилий от команды разработки, которая хочет запускать свои приложения в разных сетях. Либо можно ограничиться одной сетью, потеряв часть пользователей и рынка.

Иногда эти различия так глубоки, что для интеграции с новой платформой приходится создавать отдельное приложение-шлюз. Рассмотрим это на примере простого приложения, которое хочет узнать размер остатка на счету пользователя. И уже на этом шаге выяснится, что блокчейн сети имеют уникальные модели для хранения этой информации.

В Ethereum используется модель учетных записей (Account Model). Каждая учетная запись (смарт-контракт тоже фактически является учётной записью) имеет баланс и состояние, которые изменяются при выполнении транзакций. В этой модели информация о всех учётных записях и их балансах хранится в индексах внутри нод, и обновляется при изменений состояния глобальной сети.

В Bitcoin, напротив, используется модель UTXO (Unspent Transaction Output), где каждая транзакция ссылается на неиспользованные выходы предыдущих транзакций. Эта модель позволяет обеспечить высокую степень прозрачности в вопросе движения средств, но она весьма отличается от модели учетных записей, используемой в Ethereum.

Выше мы говорили, что Ethereum обеспечил прорыв в вопросе создания смарт-контрактов. Ethereum использует язык программирования Solidity, который специально разработан для создания смарт-контрактов на этой платформе. Solidity компилируется в байт-код, который выполняется на виртуальной машине Ethereum (EVM). Этот язык широко распространён и поддерживается большим сообществом разработчиков. На текущий момент вероятно большинство смарт-контрактов, которые являются основой децентрализованных приложений (dApps) и управляют логикой выполнения транзакций в блокчейн сетях, всё ещё пишутся на Solidity. Но новые блокчейн платформы предпочитают использовать свои языки программирования и структуры.

В Cardano использует язык программирования Plutus, основанный на функциональном языке Haskell. Plutus предоставляет мощные инструменты для создания безопасных и проверяемых смарт-контрактов, но его синтаксис и подходы существенно отличаются от Solidity.

В BitShares используется язык программирования C++, который обеспечивает высокую производительность и позволяет разработчикам создавать сложные и эффективные смарт-контракты, которые исполняются нативно. При этом, использование C++ также требует от разработчиков глубоких знаний этого языка и особенностей его применения в контексте блокчейн.

Набирающий популярность TON имеет на данный момент аж три языка для написания смарт-контрактов (FunC, Tact, Fift), и ни один из них не совместим с ранее названными\cite{label21}. При этом инструментарий для всех трёх языков сильно уступает остальным языкам, т.к. эти языки находятся на относительно ранней стадии развития.

Различия в форматах данных и языках программирования создают проблемы для интероперабельности блокчейн сетей:

\begin{enumerate}
	\item Невозможность портирования.\\
	Смарт-контракты, написанные на одном языке программирования, не могут быть легко перенесены на платформу, использующую другой язык. Некоторые прогресс в этом вопросе создают блокчейны, использующие формат WASM, который может быть получен на многих языка программирования, но на данный момент это ещё один стандарт среди многих.

	\item Сложности в проверке и аудите.\\
	Различные языки программирования и модели данных требуют различных методов проверки и аудита смарт-контрактов. Это увеличивает риск ошибок и уязвимостей, так как разработчики и аудиторы должны учитывать особенности каждой конкретной платформы.

	\item Увеличенные затраты на разработку.\\
	Разработчикам необходимо изучать новые языки программирования и адаптировать свои приложения к различным форматам данных, что увеличивает общие затраты на разработку и поддержку.
\end{enumerate}

Из-за несовместимости форматов данных и способов реализации смарт-контрактов блокчейн-сети не могут эффективно обмениваться информацией и работать вместе.

\subsection{Управление идентификацией}

Критически важным аспектом для любой блокчейн системы остаётся управление идентификацией пользователей. В каждой блокчейн сети существуют свои методы управления идентификацией и механизмы аутентификации, которые обеспечивают безопасность и анонимность пользователей. Эти различия затрудняют процесс аутентификации и верификации пользователей и транзакций при взаимодействии между различными блокчейн сетями, что создаёт барьеры для интероперабельности.

Команды, работающие над развитием блокчейн-сеть имеют свои уникальный подходы к управлению идентификацией:

\begin{itemize}
	\item В Bitcoin и его форках используется псевдонимная система, где пользователи идентифицируются по публичным ключам. Каждый адрес Bitcoin представляет собой хэш публичного ключа, и для выполнения транзакции необходимо доказательство владения соответствующим приватным ключом. Эта система обеспечивает высокий уровень анонимности, но не включает механизмов для верификации личности пользователя.

	\item В Ethereum также используется система на основе публичных и приватных ключей для управления учётными записями. Благодаря возможности создания смарт-контрактов, Ethereum позволяет внедрять более сложные схемы управления идентификацией, включая использование децентрализованных идентификаторов (DIDs) и цифровых аттестатов (VCs).

	\item В частных и консорциумных блокчейнах, таких как Hyperledger Fabric, используются разрешительные модели доступа. В этих системах участники сети идентифицируются и аутентифицируются с помощью централизованных или полуцентрализованных систем управления доступом, таких как сертификаты X.509. Это позволяет внедрять строгие меры контроля доступа и аутентификации.
\end{itemize}

Различия в методах управления идентификации приводят к ряду проблем, с которыми сталкивается пользователь:

\begin{enumerate}
	\item Аутентификация.\\
	Различия возможностей и подов к аутентификации делает сложным обеспечение безопасности при взаимодействии между сетями. В частности, система на основе публичных ключей в Bitcoin не предоставляет возможности для реализации сложных схем аутентификации, используемых в разрешительных блокчейнах, таких как Hyperledger Fabric. Это затрудняет проверку подлинности транзакций, инициированных пользователями из разных сетей.

	\item Верификация личности.\\
	В открытых блокчейнах, таких как Ethereum, пользователи могут оставаться анонимными, а в частных сетях требуется идентификация всех участников. Это создаёт проблемы при передаче данных и активов между сетями, так как нет единого стандарта для проверки личности пользователей. В результате могут возникнуть конфликты и риски безопасности.

	\item Управление доступом.\\
	В разрешительных блокчейнах часто используются сложные схемы управления доступом, основанные на ролях и атрибутах, а в публичных сетях таких механизмов нет. Это затрудняет интеграцию и совместимость систем управления доступом, что ведёт к появлению уязвимостей и несанкционированному доступу при взаимодействии сетей.

	\item Неудобство для пользователей.\\
	Пользователям бывает сложно управлять идентификацией и осуществлять транзакции между различными блокчейн сетями. От них требуются знания о разных методах аутентификации и верификации, что усложняет использование технологий и снижает их привлекательность.
\end{enumerate}

Проблема управления идентификацией в контексте интероперабельности блокчейн сетей представляет собой одну из ключевых задач, т.к. разработка решений для интеграции различных блокчейн сетей требует усилий по согласованию и преобразованию данных, а также адаптации механизмов аутентификации и верификации. Это увеличивает затраты и замедляет процесс разработки.

\subsection{Протоколы достижения консенсуса}

Блокчейн платформы используют разные механизмы консенсуса для обеспечения безопасности и подтверждения транзакций: Bitcoin использует Proof of Work (PoW), Ethereum в прошлом году перешел на Proof of Stake (PoS), а Polkadot применяет Nominated Proof of Stake (NPoS). Эти различия в протоколах консенсуса создают сложности в обеспечении взаимодействия между сетями, так как требуется согласование различных методов подтверждения транзакций и обеспечения безопасности сети.

Различия в протоколах консенсуса между блокчейн платформами создают ряд проблем для интероперабельности:
\begin{itemize}
	\item Разные механизмы консенсуса требуют различных методов подтверждения транзакций. В PoW транзакции подтверждаются майнерами, решающими криптографические задачи, тогда как в PoS и NPoS транзакции подтверждаются валидаторами на основе их доли криптовалюты или выборов номинаторов. Эти различия делают сложным разработку унифицированных протоколов для взаимодействия между сетями.

	\item Протоколы консенсуса по-разному влияют на скорость и пропускную способность сети. PoW, как правило, медленный из-за необходимости выполнения сложных вычислений, в то время как PoS и NPoS могут обеспечивать более высокую скорость обработки транзакций. Это создает дисбаланс при попытке синхронизировать транзакции между сетями, использующими различные протоколы консенсуса.

	\item В PoW майнеры получают вознаграждение за решение вычислительных задач, тогда как в PoS и NPoS валидаторы получают вознаграждение за участие в консенсусе на основе своей доли или выборов. Эти различия в экономических стимулах могут привести к конфликтам и несовместимости при попытке интеграции систем, поскольку модели вознаграждения и стимулы для участников будут различными.
\end{itemize}

Различия в протоколах консенсуса имеют значительные последствия для интероперабельности блокчейн сетей:

\begin{enumerate}
	\item Сложности интеграции.\\
	Разработка решений для взаимодействия между сетями, использующими различные протоколы консенсуса, требует сложных архитектурных изменений и согласования методов подтверждения транзакций. Это увеличивает затраты и временные рамки на разработку межсетевых решений.

	\item Безопасность.\\
	Различные протоколы консенсуса обеспечивают безопасность сети по-разному. Попытки интеграции могут привести к уязвимостям, если методы безопасности одного протокола не будут должным образом согласованы с методами другого. К примеру, атаки, эффективные против PoW, могут быть неэффективны против PoS, и наоборот.

	\item Пользовательский опыт.\\
	Для конечных пользователей различия в протоколах консенсуса могут проявляться в виде различных времён подтверждения транзакций, комиссий и безопасности. Это усложняет использование межсетевых приложений и снижает удобство и надёжность использования блокчейн технологий.
\end{enumerate}

Различия в протоколах консенсуса создают проблемы для интероперабельности блокчейн сетей, требующие комплексных и тщательно продуманных решений для их преодоления.

\subsection{Ограниченная ликвидность}

В каждой блокчейн сети создаются и поддерживаются собственные активы, которые не могут быть перемещены между сетями, но могут быть обменяны. Токены ERC-20, выпущенные на Ethereum, не могут быть напрямую использованы на остальных платформах, таких как Binance Smart Chain или Polkadot, без применения специальных мостов или обменников. Это ограничивает возможности использования активов и затрудняет их интеграцию в экосистемы других блокчейн сетей.

Активы, привязанные к одной блокчейн сети, имеют ограниченную доступность, что создаёт дополнительные барьеры для их использования:

\begin{enumerate}
	\item Ограниченные торговые площадки\\
	Активы могут быть доступны только на ограниченном числе торговых площадок, что снижает их ликвидность и увеличивает риск волатильности.

	\item Недоступность для DeFi\\
	Множество децентрализованных финансовых приложений и протоколов сосредоточены на одной или нескольких блокчейн платформах. Это ограничивает возможности использования активов на других платформах, затрудняя доступ к кредитованию, стейкингу и другим финансовым услугам.

	\item Фрагментация рынка.\\
	Рынок блокчейн активов становится фрагментированным, так как различные платформы имеют свои собственные стандарты и протоколы, что затрудняет интеграцию и использование активов в глобальном масштабе.
\end{enumerate}

Ограниченная ликвидность и доступность активов имеют серьёзные последствия как для пользователей, так и для разработчиков.

\subsection{Замедление инноваций}

Проблемы с интероперабельностью сдерживают развитие инновационных решений на базе блокчейна. Разработчикам приходится выбирать конкретную блокчейн платформу для разработки своих приложений, что ограничивает потенциал для создания кросс-платформенных приложений и снижает гибкость в использовании различных технологий. Это приводит к ряду негативных последствий для развития блокчейн экосистемы и инноваций в целом.

Этот выбор выбор накладывает ограничения на дальнейшее развитие приложения. Среди основных критериев можно выделить:

\begin{itemize}
	\item Каждая блокчейн платформа имеет свои уникальные технические особенности и ограничения. Ethereum имеет ограниченную пропускную способность и высокие комиссии за транзакции в периоды высокой нагрузки, но позволяет выполнять смарт-контракты и обладает внушительной базой пользователей. Другие платформы могут предлагать более высокую скорость и низкие комиссии, но иметь меньшую базу пользователей и поддержку.

	\item Каждая платформа имеет свою философию в вопросе хранения и представления данных. При разработке приложений, работающих с блокчейн, приходится писать адаптеры под каждую платформу.

	\item Инструменты для разработки, тестирования и развертывания смарт-контрактов и приложений часто специфичны для каждой платформы. Это означает, что разработчики, работая на одной платформе, не могут легко перенести свои знания и инструменты на другую платформу. Им приходится обучаться и работать с конкретным языком, что ограничивает их гибкость и возможности для разработки кросс-платформенных решений.
\end{itemize}

Разработчики и компании вынуждены сосредотачиваться на разработке приложений для одной платформе. Это затрудняет разработку инновационных решений, которые могли бы использовать лучшие аспекты различных блокчейн технологий. Так, децентрализованные приложения (dApps) могли бы использовать высокую скорость Solana и широкую базу пользователей Ethereum, остаются невостребованными из-за отсутствия интероперабельности. В итоге, это оказывает негативное влияние на всю экосистему:

\begin{enumerate}
	\item Ограниченные возможности для пользователей.\\
	Пользователи не могут легко использовать приложения и сервисы, работающие на разных блокчейн платформах. Это ограничивает их возможности и снижает общий опыт использования блокчейн технологий.

	\item Сложности для стартапов.\\
	Стартапы и новые проекты сталкиваются с трудностями при выборе блокчейн платформы, так как они вынуждены учитывать долгосрочные перспективы и потенциальные ограничения. Это замедляет их развитие и ограничивает их инновационный потенциал.

	\item Снижение конкурентоспособности.\\
	В условиях ограниченной интероперабельности блокчейн платформы конкурируют друг с другом, вместо того чтобы сотрудничать для создания универсальных решений. Это снижает общую конкурентоспособность блокчейн индустрии на глобальном уровне.
\end{enumerate}

Проблема интероперабельности блокчейна выражающаяся в необходимости выбора конкретной платформы и ограничении кросс-платформенной разработки, имеет далеко идущие последствия для инноваций и развития блокчейн технологий.

\section{Выводы}

Распределенные реестры и блокчейн, как их частный случай, несут огромный потенциал для различных отраслей. Безопасность, прозрачность и автоматизация, которые обеспечивают DLT, способны повысить эффективность бизнес-процессов и снизить риски. Примеры, такие как "Цифровая платформа распределенного реестра ФНС", демонстрируют практическое применение этой технологии. Смарт-контракты, работающие на блокчейне, позволяют автоматизировать сложные операции и повысить уровень доверия между участниками.

Однако, несмотря на все преимущества, существуют и серьезные ограничения. Сложность использования и недостаток знаний у обычных пользователей, необратимость транзакций, несовершенство законодательства, а также проблемы масштабируемости, связанные с ограниченной пропускной способностью блокчейнов, остаются серьезными препятствиями на пути широкого внедрения этой технологии. Одной из наиболее значительных проблем является отсутствие интероперабельности между различными блокчейн-платформами.

Проблема интероперабельности проявляется в нескольких аспектах. Разные блокчейн-сети используют различные форматы данных и языки программирования для смарт-контрактов, что приводит к несовместимости контрактов между платформами и требует дополнительных усилий для разработки межсетевых решений. Кроме того, существуют различные подходы к управлению идентификацией участников, использование разных протоколов достижения консенсуса, а также ограниченная ликвидность активов внутри их экосистем. Это замедляет инновации, поскольку разработчикам приходится выбирать конкретную платформу и сталкиваться с ограничениями кросс-платформенной разработки.

Для полноценной реализации потенциала DLT и блокчейна необходимо решить проблему интероперабельности. Это позволит создать единую экосистему, где пользователи и разработчики смогут беспрепятственно взаимодействовать, обмениваться данными и использовать преимущества различных блокчейн-платформ. Без решения этой проблемы блокчейн рискует остаться фрагментированным и недоступным для массового использования.
