\chapter*{Введение}
\addcontentsline{toc}{chapter}{Введение}

В последние годы технология распределённых реестров (DLT), произвела революцию в корпоративном управлении, предоставив новые возможности для децентрализованного управления процессами. Постепенно, эта технология находит своё место и в государственном управлении. Блокчейн, как частный случай DLT, изменил подход к хранению и передаче ценностей, обеспечив безопасность и прозрачность транзакций без необходимости в центральных посредниках. Это привело к появлению множества криптовалют и децентрализованных приложений, которые используют блокчейн для различных целей — от финансовых операций до распределённого управления сообществами.

Несмотря на стремительное развитие и внедрение блокчейн-технологий, возникает проблема фрагментации. Множество независимых блокчейнов создают ситуацию, в которой каждая сеть становится изолированным островом, и обмен активами между этими сетями представляет значительные сложности. В настоящий момент предложено несколько технологий, которые призваны решить (или частично решить) эту проблему. Каждый подход обладает своими сильными и слабыми сторонами.

Исследование методов и процессов обмена активами посредством атомарных свопов имеет критическое значение для дальнейшего развития блокчейн-индустрии. Это позволяет блокчейнам взаимодействовать друг с другом, не оставаясь разрозненными системами, что способствует росту и интеграции всей экосистемы.

Вопрос доверия всегда был ключевым аспектом в финансовых транзакциях. Традиционные системы полагаются на централизованные институты, такие как банки, для обеспечения доверия между сторонами. Блокчейн, благодаря своей децентрализованной природе и криптографическим механизмам, предлагает альтернативу, где доверие обеспечивается технологией, а не институтами. Атомарные свопы являются ярким примером такой технологии, которая устраняет необходимость в доверии, предоставляя безопасный и автоматизированный способ обмена активами.

Целью данной работы является разработать сервис, который бы позволил сторонам производить обмен активами между изолированными распределёнными реестрами без установления доверия. Достижение этой задачи возможно при условии исключения из сделки третьих сторон, использование децентрализованных сервисов не имеющих единой точки отказа или подверженной какому либо давлению, а так же полная прозрачность для сторон сделки.

Научная новизна работы заключается в реализации прототипа сервиса, который демонстрирует возможность проведения безопасного и эффективного обмена активами между двумя различными изолированными блокчейн сетями без установления доверия.

Практическая значимость работы заключается в том, что разработанный прототип может стать основой для создания полноценного сервиса, который позволит пользователям обмениваться активами между различными блокчейнами без необходимости доверять третьей стороне.

В этой работе решены следующие вопросы:
\begin{enumerate}
\item Обзор технологии распределённых реестров и её влияние на различные индустрии.
\item Описание проблемы фрагментации блокчейнов и необходимости межсетевого взаимодействия.
\item Обзор и выявление сильных/слабых сторон существующих технических решений обеспечения интероперабельности блокчейнов.
\item Детальное объяснение концепции атомарных свопов и их роли в блокчейн-экосистеме.
\item Пошаговый процесс проведения атомарного свопа между двумя участниками на тестовых сетях.
\item Анализ преимуществ и ограничений использования атомарных свопов.
\item Исследование аспектов доверия в контексте децентрализованных финансовых операций.
\item Оформление требований к поставленной задаче и проектирование архитектуры сервиса;
\item Выбор и обоснование технического стека для решения задачи построения сервиса;
\item Разработка сервиса проведения обмена и детальный разбор его исходного кода;
\item Тестирование сервиса.
\end{enumerate}

В конце работы обозначены вопросы, которые могут стать основой для дальнейших исследований.