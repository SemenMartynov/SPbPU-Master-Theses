\chapter{Обзор решений по обмену активами}

По мере распространения и роста глобальной электронной коммерции, денежных переводов между странами и международного туризма, возникает повышенная потребность в быстрых, недорогих и надежных трансграничных платежных решениях. Это создает предпосылки для модернизации существующих международных платежных систем, а также стимулирует развитие альтернативных платежных инструментов, способных удовлетворить данный спрос.

В апреле 2024 года опубликованы правки\cite{label22} в законопроект о криптовалютах, которые позволят Банку России с 1 сентября приступить к созданию экспериментальной площадки для использования криптовалют в международных расчетах.

\section{Обмен активов с привлечением посредника}

В традиционных финансовых системах обмен активами между разными сторонами осуществляется при помощи посредников, таких как банки и биржи. В контексте блокчейн-технологий роль посредника тоже существует, но ее специфика и вызовы существенно отличаются. Рассмотрим механизм обмена активами между различными блокчейнами с участием посредников, особенностям доверия между сторонами, связанным рискам.

\subsection{Типовой процесс обмена активов}

Обмен цифровыми активами с участием посредника, такого как криптовалютная биржа, включает несколько ключевых этапов. Этот процесс можно разбить на следующие шаги:

\begin{enumerate}
\item Регистрация и идентификация пользователей:
	\begin{itemize}
	\item Пользователи создают учетные записи на платформе посредника.
	\item Процесс регистрации часто включает проверку личности (KYC - Know Your Customer), что требует предоставления документов для подтверждения личности и адреса.
	\end{itemize}
\item Депозит цифровых активов:
	\begin{itemize}
	\item Пользователи переводят свои цифровые активы (например, биткоины, эфириумы) на кошельки, контролируемые посредником.
	\item Биржа обеспечивает безопасное хранение этих активов, используя горячие и холодные кошельки для управления рисками.
	\end{itemize}
\item Создание ордеров на обмен:
	\begin{itemize}
	\item Пользователи создают ордера на покупку или продажу цифровых активов, указывая количество и цену.
	\item Ордера могут быть лимитными (исполнение по указанной цене или лучше) или рыночными (немедленное исполнение по текущей рыночной цене).
	\end{itemize}
\item Совпадение ордеров:
	\begin{itemize}
	\item Торговая платформа посредника автоматически сопоставляет ордера покупателей и продавцов.
	\item При нахождении соответствующих ордеров происходит обмен активами по согласованной цене.
	\end{itemize}
\item Исполнение сделки:
	\begin{itemize}
	\item После совпадения ордеров биржа исполняет сделку, переводя активы между учетными записями участников.
	\item Обычно это включает дебетование активов со счета продавца и кредитование эквивалентной суммы активов на счет покупателя.
	\end{itemize}
\item Вывод средств:
	\begin{itemize}
	\item Пользователи могут вывести свои активы с платформы посредника на свои личные кошельки.
	\item Процесс вывода также включает проверку и может занимать некоторое время для обеспечения безопасности транзакции.
	\end{itemize}
\end{enumerate}

Этот подход заслужил популярность, благодаря своей простоте и удобству:
\begin{itemize}
\item Простота использования обеспечивается удобным пользовательским интерфейсом, который предоставляют такие платформы, как Binance, Coinbase и Kraken. Кроме того, сами платформы имеют хорошую документацию и службу пользовательской поддержки.

\item Централизованные биржи, обычно, обладают значительной ликвидностью, аккумулируя у себя средства пользователей, что обеспечивает быстрый обмен активов. Это особо значимо при обмене крупных объемов активов.

\item Репутация биржи выступает гарантом, что даже если какие-то процедуры вдруг пойдут не так, как планировалось, всегда есть группа специалистов технической поддержки, которая способна провести расследование проблемы и вернуть процесс в ожидаемое русло.
\end{itemize}

В триадицонных финансовых системах, деятельность посредников-банков регламентируется национальными законодательством и международными соглашениями. Правовая база криптовалютных сделок пока находится на начальном этапе своего развития, и проблема доверия остаётся краеугольный камень при обмене активами между блокчейнами с использованием посредника. Имеет смысл глубже изучить вопрос доверия между всеми сторонами сделки, чтобы оценить какие риски могут быть скрыты.

\subsection{Доверие пользователей к посреднику}

Очевидно, что в первую очередь необходимо доверие пользователей к посреднику. Пользователи должны быть уверены в том, что посредник будет надежно хранить их активы, корректно исполнять обмены и возвращать средства по завершении сделки.

Можно выделить три источника возникновения доверия.

\begin{enumerate}
\item Аудиты и прозрачность\\
Надежные биржи и обменники должны проводить регулярные аудиты безопасности и публиковать результаты, что способствует доверию со стороны пользователей. Если открыть площадку bestchange.com, содержащую крупнейшую базу данных посредников для обмена между пользователями, информации об аудите и прозрачности там обнаружить не получится. Но есть ссылки на сайты обменников, где иногда можно встретить заявления о пройденном аудите, но сами аудиторы так же не имеют достаточно репутации для доверия.

\item Страхование и резервы\\
Биржи могут предлагают страхование активов пользователей на случай взлома или других инцидентов. Наличие резервов для покрытия возможных потерь также повышает уровень доверия. На практике ситуация аналогична аудитом. На сайтах площадок можно встретить заявления о наличии страховок, но практическое применение этих механизмов представляется весьма сомнительным, и больше напоминают рекламный трюк.

\item Репутация площадки\\
По мере своей операционной деятельности, посредник накапливает отзывы и рекомендации от своих клиентов. Именно этот механизм лучше всего работает на площадке bestchange.com. Однако существуют механики по "накручиванию" рейтинга. Это не так опасно для бытового использования обменников (обмена не крупных сумм), но когда дело идёт о какой-то крупной сделке, это вполне может стать вектором атаки.
\end{enumerate}

Теперь рассмотри риски, которые несёт пользователь, полагаясь на доверие к бирже.

\begin{enumerate}
\item Взлом и потеря средств.\\
Централизованные биржи подвержены атакам хакеров. В случае успешного взлома пользователи могут потерять свои активы, как это произошло с Mt.Gox\cite{label23}. Эта биржа была взломана даже дважды. В 2011 году хакеры получили доступ к учетной записи аудитора биржи и смогли изменить цену биткоина до одного цента. В результате были украдены около 2000 биткоинов. В феврале 2014 года Mt. Gox объявила о потере 850,000 биткоинов, что на тот момент составляло около 7\% всех существующих биткоинов. Примерно 750000 биткоинов принадлежали клиентам биржи, а 100000 -- самой бирже. 
\item Мошенничество со стороны биржи.\\
Недобросовестные действия руководства биржи могут привести к потере средств пользователей. Это произошло в случае с QuadrigaCX, где основатель биржи обманул клиентов\cite{label24}. Расследование показало, что основатель использовал средства клиентов для личных нужд и инвестировал их в рискованные сделки. Он создавал фиктивные учетные записи на бирже, чтобы проводить фальшивые сделки и манипулировать балансами. Так же следствием было установлено, что QuadrigaCX функционировала по принципу Понци-схемы, где депозиты новых клиентов использовались для выплаты средств старым клиентам.
\item Ограничения со стороны властей.\\
В июне 2021 года Управление по финансовому регулированию и надзору Великобритании выпустило уведомление, что Binance Markets Limited, британское подразделение Binance, не имеет разрешения на проведение регулируемой деятельности в стране\cite{label25}. FCA потребовало от Binance прекратить все рекламные и финансовые операции в Великобритании, что фактически означало запрет на работу биржи в регионе. Пользователи из Великобритании столкнулись с ограничением доступа к некоторым сервисам и продуктам Binance, что вызвало беспокойство и неудобства.
\end{enumerate}

Можно видеть, что практически все выявленные риски уже реализовывались на практике, и пользователь не имеет возможности влиять на этот процесс.

\subsection{Доверие посредника к пользователям}

Для успешного функционирования криптовалютных бирж и обменных платформ необходимо, чтобы не только пользователи доверяли посредникам, но и посредники могли доверять своим пользователям. Это доверие основывается на ряде процедур и механизмов, которые позволяют посреднику оценивать и управлять рисками, связанными с деятельностью пользователей.

\begin{itemize}
\item Процедуры KYC (Know Your Customer): Посредники требуют от пользователей прохождения процедуры идентификации, которая включает сбор личных данных, таких как имя, адрес проживания, дата рождения, и иные документы, удостоверяющие личность. Это помогает убедиться, что пользователь является реальным человеком и не занимается мошеннической деятельностью.

\item AML (Anti-Money Laundering): Посредники внедряют меры по предотвращению отмывания денег, такие как мониторинг транзакций на предмет подозрительной активности, регулярные проверки и анализ рисков. Эти меры направлены на выявление и предотвращение использования платформы для незаконной деятельности.

\item Системы управления рисками: Посредники разрабатывают и внедряют системы для оценки рисков, связанных с пользователями. Это может включать в себя анализ транзакционной истории, проверку на санкции и черные списки, а также использование алгоритмов машинного обучения для выявления аномалий в поведении пользователей.
\end{itemize}

Рассмотрим риски, которые несёт посредник, излишне доверяющий пользователю.

Наиболее распространенная схема мошенничества основана на попытках обойти процедуру KYC. Пользователи могут попытаться использовать поддельные документы для прохождения KYC или создавая фальшивые учетные записи. Мошеннические схемы могут включать в себя фишинговые атаки, использование украденных данных и попытки взлома платформы.

Далее, поддельные учётные записи могут использоваться в схемах по отмыванию денег. Если пользователь проводит крупные транзакции, не имеющие явного экономического обоснования, это может привлечь внимание регуляторов и привести к правовым последствиям для посредника.

Посредники рискуют быть использованными для финансирования преступной деятельности. В 2017 году основатель биржи BTC-e, Александр Винник, был арестован по обвинению в отмывании более 4 миллиардов долларов через платформу\cite{label26}. BTC-e использовалась для перевода средств, связанных с киберпреступлениями, что привело к серьезным юридическим последствиям и закрытию биржи.

Любой инцидент, связанный с незаконной деятельностью пользователей, может серьезно подорвать репутацию посредника, что приведет к потере доверия со стороны других клиентов и партнеров.

Доверие между сторонами приводит к повышению рисков, которые бизнес обычно закладывает в комиссии по операциям. Более предпочтительным способом проведения сделки оказывается такой механизм, в котором необходимость доверия между сторонами сведена к минимуму.

\section{Атомарные свопы}

Атомарные свопы (atomic swaps) представляют собой технологию, позволяющую двум сторонам обмениваться криптовалютами из разных блокчейнов без необходимости установления доверия. Результат достигается с помощью контрактов с временной блокировкой и хешированием (Hashed TimeLock Contracts, HTLC). Это криптографическая схема, впервые представленная в 2016 году \cite{label14}, подтверждает легитимность действий участников, но при этом сами действия разнесены по времени.

\subsection{Требования к блочейну}

Как следует из названия, HTLC комбинирует два механизма блокировки транзакции:

\begin{itemize}
\item Механизм хеш-блокировки (hashlock) позволяет блокировать смарт-контракт с помощью уникального криптографического ключа, который может быть сгенерирован только отправителем криптовалюты в сделке. Этот уникальный ключ, представляющий собой фрагмент данных, гарантирует, что обмен активами будет завершен только после согласия и подтверждения операции обеими сторонами сделки. Иными словами, хеш-блокировка обеспечивает, что совершение свопа возможно только при наличии согласованных условий обмена от участвующих сторон, предотвращая возможность односторонних действий.
\item Механизм Timelock в атомарном свопе работает как защитный механизм, устанавливающий временные рамки для завершения обмена. Если обе стороны подтверждают транзакцию в заданный период, обмен происходит успешно. В противном случае сделка аннулируется, а криптовалюта возвращается первоначальным владельцам. Механизм timelock выступает гарантом безопасности, предотвращая бесконечную блокировку средств и обеспечивая возврат в случае невыполнения условий сделки одной из сторон.
\end{itemize}

Такой подход обеспечивает возможность отмены операции, если одна из сторон не выполняет свою часть действий.

Для реализации атомарного свопа между парой блокчейнов необходимо:

\begin{itemize}
\item Чтобы оба блокчейна поддерживали HTLC. Это не обязательно должно быть реализовано на уровне смарт-контракта, к примеру BitShares имеет реализацию непосредственно в ядре.
\item Для предотвращения мошенничества, блокчейн не должен допускать изменение идентификатора транзакции, поскольку это может ввести в заблуждение контрагента по обмену.
\item Кроме того, для успешного совершения обмена между двумя разными блокчейнами необходимо, чтобы оба блокченйна обладали возможностью использования одной и той же криптографической хеш-функции, например, SHA-256 (реже используется RIPEMD-160). Данная процедура требуется для обеспечения корректной работы смарт-контракта в ситуациях, когда происходит формирование хеш-значения от секретного числа.
\end{itemize}

Вопреки распространенному мнению, технология Lightning Network не является обязательным требованием для успешного проведения атомарных свопов. Хотя атомарные свопы могут осуществляться и без Lightning Network, эта технология способна значительно оптимизировать процесс обмена, делая его более удобным, быстрым и эффективным. Как и в основе атомарных свопов, в основе Lightning Network лежат смарт-контракты с использованием хеш-блокировки и временной блокировки. Если атомарные свопы обеспечивают взаимодействие между различными блокчейнами, то Lightning Network фокусируется на соединении платежных каналов\cite{label27}.

\subsection{Процесс атомарного свопа}

Атомарный своп включает несколько ключевых шагов, каждый из которых критически важен для успешного выполнения обмена. Рассмотрим эти шаги более подробно.

\begin{enumerate}
\item Инициация сделки:
	\begin{itemize}
	\item Алиса и Боб решают обменять криптовалюты между блокчейнами A и B.
	\item Алиса генерирует случайный секрет $ S $ и создает его хеш $ H(S) $. Этот хеш будет использоваться для создания HTLC.
	\end{itemize}
\item Создание HTLC на блокчейне A:
	\begin{itemize}
	\item Алиса создает контракт HTLC на блокчейне A и депонирует в него свои средства. Контракт содержит Хеш $ H(S) $ и время блокировки $ T_A $, по истечении которого Алиса может вернуть свои средства, если обмен не завершится успешно.
    \item Контракт становится доступным для всех участников сети, включая Боба.
	\end{itemize}
\item Создание HTLC на блокчейне B:
	\begin{itemize}
	\item Узнав хеш $ H(S) $, Боб создает аналогичный HTLC на блокчейне B и депонирует свои средства. Контракт содержит тот же хеш $ H(S) $ и время блокировки $ T_B $, которое меньше $ T_A $, чтобы обеспечить возврат средств Бобу до того, как Алиса сможет вернуть свои.
	\end{itemize}
\item Проверка и подтверждение:
	\begin{itemize}
	\item Алиса проверяет контракт Боба на блокчейне B и убеждается, что все условия соблюдены.
	\end{itemize}
\item Предъявление секрета и завершение свопа:
	\begin{itemize}
	\item Алиса предъявляет секрет $ S $ в контракте Боба на блокчейне B и забирает свои средства.
	\item Узнав секрет $ S $ из публичного контракта на блокчейне B, Боб предъявляет его в контракте Алисы на блокчейне A и забирает свои средства.
	\end{itemize}
\end{enumerate}

Процесс атомарного свопа разработан так, чтобы обеспечить безопасность и надежность для обеих сторон. Если Боб не создаст HTLC на блокчейне B, Алиса получит возможность вернуть свои средства после истечения времени блокировки $ T_A $. Если Алиса не предъявит секрет $ S $ и не заберет средства из HTLC на блокчейне B, Боб может вернуть свои средства после истечения времени блокировки $ T_B $.

\subsection{Возможности и ограничения}

Атомарные свопы остаются сложными для массового использования, но эта технология приводит к существенным изменениям в отношении функциональной совместимости блокчейнов и децентрализованного обмена. Перечислим сильные сторонами этой технологии:

\begin{enumerate}
\item Полная децентрализация.\\
Атомарные свопы предоставляют трейдерам возможность полностью децентрализованного обмена, исключая необходимость в централизованных платформах (CEX) и ликвидностных пулах. Пиринговый трейдинг позволяет пользователям сохранять контроль над своими средствами и быть независимыми от централизованных посредников.

\item Безопасность транзакций.\\
Атомарные свопы организованы таким образом, что либо обмен происходит и обе стороны получают средства, либо ничего не происходит, и обе стороны сохраняют свои начальные средства (за вычетом небольшой комиссии для тейкера). Такой механизм обеспечивает максимальную безопасность цифровых активов, минимизируя риск потерь и делая транзакции дешевле по сравнению с централизованными биржами. Во время процесса атомарного свопа пользователь никогда не утрачивают контроль над своими монетами и токенами, исключая необходимость передачи контроля третьей стороне.

\item Высокая скорость транзакций\\
Атомарные свопы позволяют проводить обмены значительно быстрее, с временем обработки, равным скорости одной транзакции.
\end{enumerate}

Основным ограничивающим фактором для повсеместного использования атомарных свапов остаётся ограниченная поддержка HTLC блокчейн-сетями. Кроме того, среди ограничений можно так же отметить:

\begin{enumerate}
\item Слабая конфиденциальность.\\
Атомарные свопы не требуют процедуры KYC и изначально обеспечивают приватность сделок. Но транзакции, выполненные через публичные блокчейны, такие как Bitcoin и Litecoin, могут быть отслежены. Для повышения конфиденциальности участники могут применять технологии смешивания монет или использовать криптовалюты, ориентированные на конфиденциальность, такие как Monero (XMR) или Zcash (ZEC).

\item Ограничения по типам активов.\\
Атомарные свопы могут использоваться только для обмена цифровыми активами. Невозможно обменять цифровые активы на фиатные валюты, такие как рубли, из-за невозможности запуска HTLC контрактов на физические активы. Ситуация должна измениться с развитием цифрового рубля.

\item Ликвидность.\\
При необходимости проведения крупной сделки может потребоваться значительное время для поиска встречной стороны. В таких случаях вероятно потребуется разбить одну крупную заявку на несколько меньших, что доставит некоторые неудобства и замедлит процесс обмена.
\end{enumerate}

Атомарные свопы представляют собой инновационный механизм обмена цифровыми активами, который повышает безопасность и скорость транзакций, сохраняя при этом децентрализованный характер операций. Однако существуют определенные ограничения и вызовы, связанные с их применением, и их имеет смысл учитывать при внедрении атомарных свопов в свои процессы.

\section{Мосты}

Под мостом (так же кроссчейн-мост, Cross-chain Bridge) чаще всего подразумевается использование управляющих смарт-контрактов и сервиса-оракула, который мониторит транзакции, прослушивает события в управляющем смарт-контракте. Мосты в блокчейн можно разделить на централизованные и децентрализованные.

Централизованные мосты управляются одной или несколькими доверенными сторонами, которые обеспечивают перемещение активов между блокчейнами. Доверенной стороной может быть не обязательно какое-то третье лицо, эту роль вполне способен выполнить и один из участников сделки. Это снижает список рисков, но не позволяет полностью уйти от необходимости доверия.

Децентрализованные мосты (или trustless bridges) работают без участия центрального посредника. Они используют смарт-контракты и децентрализованные системы консенсуса для обеспечения безопасности и надежности транзакций. Эти мосты устраняют необходимость доверия к третьей стороне, что делает их более безопасными и прозрачными.

Когда мы говорим про обмен активами, мы не имеем в виду прямой перенос, т.к. это невозможно. Но в случае с мостами, существует два способа передачи активов: так называемые обернутые токены (wrapped token) и пулы ликвидности.

\subsection{Метод обернутого токена}

Обернутые токены представляют собой токены, выпущенные на одном блокчейне, которые отражают стоимость активов, хранящихся на другом блокчейне. К примеру, в сети Ethereum выпускается wrapped-BTC, который по стоимости эквивалентен исходному биткоину. Для создания обёрнутых монет существует три вида транзакций: lock-and-mint, burn-and-release и burn-and-mint.

Транзакция lock-and-mint подразумевает блокировку оригинального актива в исходной сети и выпуск эквивалентного обернутого токена в целевой сети. Процесс начинается с того, что пользователь отправляет свой актив на определенный адрес или смарт-контракт в исходной сети. Этот актив блокируется, чтобы гарантировать его недоступность для других операций. После подтверждения блокировки в целевой сети выпускается обернутый токен, который представляет собой эквивалент заблокированного актива. Этот токен может быть использован в целевой сети как полноценный актив, предоставляя пользователям доступ к функциональности другой блокчейн-сети.

Транзакция burn-and-release используется для обратного процесса, когда пользователь желает вернуть оригинальный актив из целевой сети в исходную. Для этого обернутый токен в целевой сети отправляется на смарт-контракт, который его сжигает, то есть удаляет из обращения. После подтверждения операции в исходной сети разблокируется эквивалентный оригинальный актив, который становится доступным для пользователя. Этот процесс обеспечивает баланс и поддерживает эквивалентное количество активов в обеих сетях.

Транзакция burn-and-mint представляет собой сочетание элементов двух предыдущих стратегий, но используется в некоторых специализированных сценариях. В этом случае, чтобы переместить активы между сетями, пользователь сжигает обернутый токен в одной сети, что инициирует выпуск нового токена в другой сети. Такая стратегия применяется в системах, где требуется поддержание строгой эквивалентности активов, но при этом существует необходимость в их постоянном перемещении между различными блокчейнами.

Рассмотрим сценарий использования кросс-чейн моста. Алиса хочет обменять свой BTC на ERC20 wrapped-BTC, чтобы воспользоваться сервисами сети Ethereum. 

\begin{enumerate}
\item Инициация транзакции. Алиса решает обменять свой Bitcoin на wrapped-BTC и инициирует транзакцию через интерфейс кросс-чейн моста. Она указывает количество BTC, которое хочет обменять, и адрес Ethereum кошелька, куда должен быть отправлен wrapped-BTC.
\item Отправка BTC. Алиса отправляет указанное количество BTC на адрес, управляемый кросс-чейн мостом. Этот адрес может быть смарт-контрактом или мультиподписью (multisig) кошельком, который блокирует активы.
\item Подтверждение блокировки. Когда транзакция BTC подтверждена в сети Bitcoin, кросс-чейн мост фиксирует факт получения и блокировки BTC. Эти данные передаются в сеть Ethereum.
\item Выпуск wrapped-BTC. На основании подтверждения блокировки BTC, смарт-контракт на блокчейне Ethereum выпускает эквивалентное количество wrapped-BTC и отправляет их на указанный адрес Алисы в сети Ethereum.
\item Использование wrapped-BTC в сети Ethereum. После зачисления wrapped-BTC на свой счёт, Алиса получает возможность использовать их в экосистеме Ethereum, включая торговлю на децентрализованных биржах (DEX), участие в протоколах DeFi, или любые другие действия, доступные для токенов стандарта ERC20.
\end{enumerate}

Обратная операция так же выглядит не сложно. Когда Алиса хочет обменять свои wrapped-BTC обратно на оригинальные BTC, она выполняет следующие шаги:

\begin{enumerate}
\item Инициация обратной транзакции. Алиса решает вернуть свой wrapped-BTC обратно в Bitcoin и инициирует обратную транзакцию через интерфейс кросс-чейн моста, указывая количество wrapped-BTC для обмена и адрес, на который должен быть отправлен BTC.
\item Отправка wrapped-BTC. Алиса отправляет wrapped-BTC на адрес смарт-контракта в сети Ethereum, который отвечает за сжигание токенов. Это действие инициализирует процесс разблокировки оригинальных BTC.
\item Подтверждение сжигания. Смарт-контракт сжигает wrapped-BTC, уменьшив общее количество токенов в обращении, и фиксирует факт сжигания. Эти данные передаются в сеть Bitcoin.
\item Разблокировка BTC. На основании подтверждения сжигания wrapped-BTC кросс-чейн мост разблокирует эквивалентное количество BTC и отправляет их на указанный адрес Алисы в сети Bitcoin.
\end{enumerate}

Используя кросс-чейн мост, Алиса может безопасно обменивать свои BTC на wrapped-BTC в сети Ethereum и обратно, используя преимущества обеих блокчейн-экосистем. Даже если мост находится под управлением централизованного посредника, смарт-контракты могут быть составлены так, что процесс обменна остаётся гарантированно безопасным и надежным.

\subsection{Метод пула ликвидности}

Метод пула ликвидности как подход, позволяющий пользователям обмениваться активами, пришёл в мир блокчейн из традионных финансов. Этот метод основан на создании децентрализованных пулов активов, доступных для обмена между различными блокчейнами. В отличие от предыдущего метода обёрнутого токена, пулы ликвидности предлагают пользователям возможность напрямую обменивать свои токены через ликвидные резервы, которые поддерживаются поставщиками ликвидности. Эти пулы состоят из токенов, внесенных различными участниками, которые предоставляют свои активы в обмен на долю от комиссий за транзакции и возможные вознаграждения.

Основной принцип работы пула ликвидности состоит в том, что пользователи, желающие обменять один актив на другой, могут делать это через смарт-контракты, которые управляют резервами ликвидности. Когда пользователь вносит свои токены в пул, смарт-контракт автоматически выполняет обмен, используя резервные активы. Это устраняет необходимость в сложных процессах блокировки и выпуска токенов, предлагая более прямолинейный и эффективный способ обмена.

Условием работоспособности этого метода является наличие поставщиков ликвидности (LPs), которые вносят свои активы в пулы в обмен на ликвидные токены (LP-токены). Для этого, поставщик ликвидности сначала должен выбрать платформу и пул ликвидности, в который он хочет внести свои активы. Например, на платформе Uniswap поставщик может выбрать пул, состоящий из пары токенов, таких как ETH/USDT. LP решает, сколько каждого из этих токенов он хочет внести. Стоит отметить, что большинство пулов требуют внесения активов в равных пропорциях по стоимости.

LP отправляет свои токены в смарт-контракт, управляющий пулом ликвидности. Этот смарт-контракт автоматически обновляет резервы пула и выпускает LP-токены, которые представляют долю LP в пуле. LP-токены подтверждают право на часть активов пула и соответствующую долю от комиссий за транзакции, совершаемые в пуле.

После внесения активов в пул, поставщик ликвидности получает LP-токены. Количество LP-токенов пропорционально доле, внесенной LP в общий резерв пула. Токены могут быть использованы для подтверждения права на изъятие активов из пула вместе с накопленными комиссиями.

Мотивация поставщиков ликвидности заключается в возможности заработать на комиссиях за транзакции, происходящие в пуле. Каждый раз, когда кто-то обменивает токены через этот пул, взимается комиссия, которая распределяется между всеми поставщиками ликвидности пропорционально их доле в пуле, обеспечивая для них стабильный пассивный доход.

Кроме того, поставщики ликвидности могут получать вознаграждения в виде дополнительных токенов. Некоторые платформы, такие как SushiSwap, предлагают вознаграждения за участие в пулах ликвидности в виде их собственных токенов управления (В частности SUSHI). Это создает дополнительный стимул для внесения активов в пулы ликвидности.

LP-токены могут быть использованы несколькими способами:
\begin{enumerate}
\item Изъятие активов из пула.\\
LP может в любое время вернуть свои LP-токены смарт-контракту, в результате произойдёт выемка внесенных активов вместе с долей от накопленных комиссий.
\item Торговля LP-токенами.\\
LP-токены могут быть проданы или переданы другим пользователям. Это полезно, если LP хочет выйти из пула ликвидности но без изъятия активов напрямую.
\item Использование в других протоколах DeFi.\\
LP-токены могут быть использованы в других децентрализованных финансовых протоколах для получения дополнительных доходов. Их можно использовать в качестве залога для получения кредитов или для участия в фарминге доходности (yield farming), где они могут приносить дополнительные вознаграждения.
\end{enumerate}

Процесс внесения активов в пулы ликвидности и использование LP-токенов предоставляет поставщикам ликвидности различные возможности для получения дохода и использования своих активов в экосистеме DeFi. Это стимулирует участие в пулах ликвидности и способствует повышению ликвидности и эффективности децентрализованных бирж и финансовых приложений. Участие в пулах ликвидности тоже сопряжено с рисками. Один из ключевых рисков -- так называемые "непостоянные потери" (impermanent loss). Они возникают при изменении цены активов в пуле относительно рыночной цены. Такие убытки называются временными, т.к. несоответствие цены со временем должно выправиться, а убыток станет реальным только при выводе средств из пула. Но на какое-то время пользователь оказываются запертым в пуле, пока ждёт восстановления цен.

Для пользователя, который хочет совершить обмен через пул, всё происходит просто и прозрачно.

Пользователь выбирает платформу, которая поддерживает обмен интересующих его токенов. На платформе должен быть пул ликвидности, который содержит необходимую пару. В интерфейсе обмена нужно указать сколько монет нужно обменять. Платформа автоматически вычисляет количество монет, которое пользователь получит в обмен, на основе текущего обменного курса и ликвидности пула, а так же рассчитывает все комиссии, включая стоимость газа. Пользователю останется только подтвердить обмен и платформа отправит транзакцию в блокчейн.

пользователю стоит обратить внимание на ликвидность пула. При обмене через пулы ликвидности с низкой ликвидностью или при больших объемах обмена появляется риск проскальзывания (slippage), когда конечный обменный курс отличается от ожидаемого в момент подтверждения транзакции.

Пулы ликвидности это простой, удобный, быстрый и надёжный инструмент, который хорошо работает при небольших объёмах обмена и на высоко ликвидных парах активов. В остальных случаях, итоговая цена проведения обмена может произойти по менее выгодному для пользователя курсу.

\section{Релейный блокчейн}

Основная идея релейного блокчейна (Relay Chain) заключается в создании единой, централизованной цепочки, которая координирует и валидирует транзакции, происходящие на множестве параллельных блокчейнов (парачейнов). Это позволяет создать масштабируемую и гибкую экосистему, в которой каждая парачейн может иметь собственную логику и специфику, но при этом все они будут интегрированы и смогут взаимодействовать друг с другом через релейный блокчейн. Такой подход решает одну из главных проблем блокчейнов первого и второго поколений, связанных с ограниченной масштабируемостью и недостатком межсетевого взаимодействия.

Релейный блокчейн обеспечивает два ключевых аспекта: безопасность и консенсус. Безопасность достигается за счет использования механизма общей безопасности, где валидаторы релейного блокчейна подтверждают транзакции всех парачейнов. Это означает, что парачейны не нуждаются в собственной сети валидаторов, что значительно упрощает их развертывание и управление. Консенсус достигается благодаря использованию таких механизмов, как Proof of Stake (PoS) или прочих алгоритмов достижения консенсуса, что обеспечивает высокую скорость и эффективность подтверждения транзакций.

Концепция релейного блокчейна была популяризирована проектом Polkadot, разработанным командой Parity Technologies во главе с Гэвином Вудом, сооснователем Ethereum. Релейный блокчейн играет центральную роль в обеспечении безопасности, консенсуса и взаимодействия между различными парачейнами (parachains), работающими в экосистеме. Блокчейн Polkadot использует механизм Nominated Proof of Stake (NPoS) для выбора валидаторов и достижения консенсуса. В системе Polkadot релейный блокчейн играет роль центральной координирующей цепочки, тогда как парачейны могут выполнять специфические задачи, такие как выполнение умных контрактов, управление идентификацией или создание децентрализованных приложений (dApps). Такая архитектура позволяет разделить нагрузку и повысить общую производительность сети.

Kusama, несмотря на роль тестовой площадки для Polkadot, функционирует как автономная и полноценная блокчейн-сеть. Kusama позволяет разработчикам тестировать и развертывать свои парачейны в реальных условиях перед их интеграцией в основную сеть Polkadot.

Релейный блокчейн также важен для реализации концепции интероперабельности, позволяя различным блокчейнам взаимодействовать и обмениваться данными. Это достигается через механизмы кроссчейн-коммуникации, такие как межцепочные сообщения (Cross-Chain Messaging). Например, в сети Polkadot парачейны могут обмениваться информацией и активами через релейный блокчейн, что открывает новые возможности для создания комплексных децентрализованных приложений и сервисов.

Проект Cosmos нацелен на создание интероперабельной экосистемы, в которой различные блокчейны могут взаимодействовать и обмениваться данными. Разработанный командой Tendermint, Cosmos использует уникальный протокол Inter-Blockchain Communication (IBC), который позволяет блокчейнам в сети Cosmos обмениваться информацией безопасно и эффективно. Cosmos построен на базе Tendermint Core, консенсусного алгоритма на основе PoS (Proof of Stake), обеспечивающего высокую производительность и низкую задержку транзакций. Сеть Cosmos Hub выступила первым блокчейном в экосистеме, играет роль релейной цепочки, координируя взаимодействие между различными зонами (zones), аналогичными парачейнам в Polkadot.

Cosmos разработан с акцентом на модульность и настраиваемость, что делает его очень гибким инструментом. С использованием Cosmos SDK разработчики могут легко создавать и запускать собственные блокчейны, адаптированные под конкретные задачи и требования. В отличие от Polkadot, где парачейны должны регистрироваться и быть частью единой экосистемы, в Cosmos каждая зона может функционировать как автономный блокчейн, подключаясь к Cosmos Hub для межцепочных взаимодействий. Это обеспечивает гибкость и возможность для разработчиков использовать различные консенсусные механизмы и правила управления, адаптированные к их потребностям.

Сравнение Cosmos и Polkadot выявляет несколько ключевых различий и сходств. Обе платформы стремятся решить проблему интероперабельности и масштабируемости блокчейн-сетей, однако подходы у них различаются. Polkadot использует единую релейную цепочку для координации парачейнов и обеспечения общей безопасности, тогда как Cosmos позволяет блокчейнам быть более независимыми, соединяя их через Cosmos Hub и протокол IBC. В Polkadot валидаторы подтверждают транзакции всех парачейнов, что централизует безопасность. В Cosmos безопасность каждой зоны обеспечивается ее собственными валидаторами, что может привести к различиям в уровне безопасности между зонами. Оба проекта используют PoS для достижения консенсуса, но Cosmos акцентирует внимание на модульности и гибкости, предоставляя разработчикам инструменты для создания кастомизированных блокчейнов.

Один из основных минусов заключается в сложности и стоимости подключения новых парачейнов. В экосистеме Polkadot, например, парачейны должны проходить процесс регистрации и обеспечения слотов, что может быть дорогостоящим и конкурентным процессом. Это создает барьеры для входа для новых проектов и ограничивает разнообразие и инновации внутри экосистемы.

Зависимость от релейной цепочки создает определенную централизацию, что также можно считать недостатком. Хотя Polkadot и другие подобные системы используют децентрализованные механизмы консенсуса, сама концепция единой релейной цепочки подразумевает определенную степень централизации. В условиях, где важна максимальная децентрализация и автономия каждой отдельной сети, подобный подход не применим. Любая централизация ведёт в вопросу доверия и злоупотреблений. Формально,все операции выполняются под контролем верифицированной логики, но централизация допускает возможность внесения регрессий в эту логику и фактическое поведение будет отличаться от ожидаемого.

Наконец, сложность архитектуры релейных блокчейнов приводит к трудностям в управлении и обновлении системы. Координация большого числа парачейнов требует сложных протоколов и увеличивает риск ошибок или уязвимостей. Управление и обеспечение безопасности в такой комплексной системе требует заметных ресурсов и усилий.

Если посмотреть на задачу обмена активами между независимыми блокчейнами, то становится заметно, что релейные блокчейны, ориентированы на взаимодействие между парачейнами внутри своей экосистемы. Для обмена активами между блокчейнами, не являющимися частью одной и той же релейной цепочки, требуется использование мостов или других методов межцепочного взаимодействия. Polkadot и Cosmos предоставляют решения для таких сценариев через свои механизмы межцепочной коммуникации и мостов, но это все равно предполагает, что обе сети поддерживают такие взаимодействия. На сегодняшний момент нет заметной активности со стороны независимых блокчейнов по интеграции у себя таких механизмов, а значит это решение не может быть универсальным.

Технологическая основа релейных блокчейнов требует тщательной проработки протоколов и архитектуры для обеспечения безопасности и эффективности. Релейный блокчейн позволяет обрабатывать большое количество транзакций и активов между блокчейнами, обеспечивая высокую пропускную способность и эффективность. Но решением проблемы интероперабельности релейные блокчейны не являются из-за ограничения "общего знаменателя": участвующие в обмене блокчейны должны поддерживать средства межцепочной коммуникации, которые предлагает платформа. Другими словами, все блокчейны должны быть собраны на одном фреймворке, что не всегда возможно. Особенно это требование сложно выполнить для уже работающих блокчейнов с историей. Так же специалисты выражают беспокойство по вопросу централизации релейных блокчейнов.

\section{Выводы}

Растущий спрос на быстрые, дешёвые и надежные трансграничные платежи стимулирует развитие новых решений для обмена активами между блокчейнами. Обмен активами между независимыми блокчейнами может осуществляться различными методами, каждый из которых имеет свои особенности и области применения.

Централизованные биржи (CEX) предоставляют простой и удобный способ обмена активами, требуя доверия к бирже как посреднику. Пользователи депонируют свои активы на бирже, совершают сделки и выводят полученные активы. Основные преимущества CEX включают высокую ликвидность, быстрые транзакции и поддержку множества криптовалют. Однако, они подвержены рискам взлома, мошенничества и требуют доверия к бирже, что может быть неприемлемо для пользователей, стремящихся к полной децентрализации.

Атомарные свопы с использованием HTLC (Hash Time-Locked Contracts) представляют собой децентрализованный способ обмена активами без необходимости доверять третьей стороне. Они обеспечивают безопасность и гарантии выполнения обмена либо полностью, либо не выполняется вообще. Это делает атомарные свопы идеальным решением для пользователей, стремящихся к максимальной безопасности и децентрализации. Однако, такие свопы могут быть сложны в реализации и ограничены в поддержке сложных сценариев обмена, а также требуют, чтобы обе участвующие блокчейны поддерживали необходимые функции смарт-контрактов.

Мосты, использующие обернутые монеты и пулы ликвидности, предлагают гибкость и позволяют обменивать активы между различными блокчейнами путем создания обернутых токенов, которые представляют собой эквиваленты оригинальных активов на другой блокчейн. Пулы ликвидности обеспечивают ликвидность для таких обменов, что позволяет пользователям быстро и эффективно обменивать активы. Мосты с обернутыми токенами позволяют поддерживать взаимодействие между различными блокчейнами, но требуют доверия к мостам и могут быть подвержены уязвимостям. Они особенно полезны в экосистемах с высокой степенью децентрализации, где необходима гибкость и возможность обмена активами между многочисленными блокчейнами.

Релейные блокчейны, такие как Polkadot, предоставляют интегрированное и безопасное решение для обмена активами внутри своей экосистемы. Парачейны могут обмениваться активами и данными напрямую через релейную цепочку, что обеспечивает надежность и быструю валидацию транзакций. Релейные блокчейны подходят для крупных и комплексных экосистем, где важна общая безопасность и согласованность. Однако, обмен активами между независимыми блокчейнами вне такой экосистемы может потребовать использования мостов или других механизмов межцепочного взаимодействия.

В зависимости от конкретных потребностей и контекста использования, выбор между этими методами может варьироваться. Централизованные биржи подходят для пользователей, которым важна простота и высокая ликвидность, несмотря на риски централизованного управления. Атомарные свопы предпочтительны для тех, кто стремится к максимальной безопасности и децентрализации. Мосты с обернутыми монетами и пулами ликвидности обеспечивают гибкость и межцепочное взаимодействие, требуя при этом некоторого уровня доверия. Релейные блокчейны идеальны для интегрированных экосистем, обеспечивая надежность и согласованность в пределах своей сети.
