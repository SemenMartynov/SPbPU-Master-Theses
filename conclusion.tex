\chapter*{Заключение}
\addcontentsline{toc}{chapter}{Заключение}

В первом разделе данной работы проведено исследование технологии распределенных реестров, включая технологию блокчейн, как её частный случай. Проанализировано влияние DLT на различные отрасли, выделены её ключевые преимущества – безопасность, прозрачность и автоматизация. Особое внимание уделено проблеме интероперабельности блокчейн сетей, которая ограничивает взаимодействие между различными блокчейн платформами.

Во втором разделе рассмотрены существующие решения для обмена активами между блокчейнами, такие как централизованные биржи, атомарные свопы, мосты с использованием обернутых монет и пулов ликвидности, а также релейные блокчейны. Выявлены их сильные и слабые стороны, а также области применения. Отдельное внимание уделено вопросу доверия между сторонами. Проведено исследование возникновения этого доверия и возможные последствия для реальных проектов.

В последнем разделе была поставлена задача -- разработка сервиса для проведения обмена активами между изолированными распределенными реестрами без установления доверия. После анализа существующих подходов был сделан вывод, что механизм атомарных свопов с использованием HTLC является наиболее эффективным и безопасным.

Основные результаты работы:
\begin{itemize}
\item Разработан прототип сервиса обмена активами, использующий механизм атомарных свопов с HTLC. Сервис реализован с использованием React для создания пользовательского интерфейса, Ethereum для развертывания смарт-контрактов HTLC и BitShares для проведения обмена активами без использования виртуальной машины.

\item Проведено модульное тестирование разработанного сервиса, подтверждающее корректность работы отдельных компонентов.

\item Выполнено системное тестирование в тестовых сетях Ethereum (Ganache) и BitShares, демонстрирующее работоспособность сервиса в реальных условиях.
\end{itemize}

Возможные направления дальнейших исследований:
\begin{itemize}
\item Улучшение пользовательского интерфейса для упрощения взаимодействия с сервисом, особенно для пользователей BitShares и других блокчейн-проектов, не имеющих поддержки в популярных криптокошельках.

\item Интеграция с другими блокчейн платформами, расширяющая функциональность сервиса и увеличивающая количество доступных для обмена активов.

\item Исследование возможности реализации механизма HTLC для других популярных блокчейнов, таких как Bitcoin и Litecoin.

\item Разработка децентрализованного механизма хранения приватных ключей для повышения безопасности и удобства использования сервиса.
\end{itemize}

В заключение, разработанный прототип сервиса обмена активами демонстрирует потенциал механизма атомарных свопов с HTLC для решения проблемы интероперабельности блокчейн сетей. Дальнейшие исследования и разработки в этом направлении могут привести к созданию более совершенных и доступных инструментов для обмена активами между различными блокчейн платформами, способствуя развитию децентрализованных финансовых систем и укреплению доверия к блокчейн технологиям.