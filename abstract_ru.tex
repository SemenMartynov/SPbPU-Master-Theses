\begin{center}
\textbf{РЕФЕРАТ}
\end{center}
\vspace{4pt}

На 127 с., 6 рисунков

\noindent ОБМЕН АКТИВАМИ, РАСПРЕДЕЛЁННЫЕ РЕЕСТРЫ, БЛОКЧЕЙН, ДОВЕРИЕ, ИНТЕРОПЕРАБЕЛЬНОСТЬ, АТОМАРНЫЙ СВОП, HTLC, ETHEREUM, BITSHARES, REACT.

\vspace{4pt}

Тема выпускной квалификационной работы: "Разработка сервиса для проведения обмена активами между изолированными распределёнными реестрами без установления доверия".

В рамках работы проведено исследование технологии распределенных реестров (DLT), включая технологию блокчейн, и изучены различные механизмы решения проблемы интеропребельности блокчейнов в части обмена активами между различными блокчейн-сетями.

Результаты ВКР:
\begin{enumerate}
\item Изучены существующие методы обмена активами и выявлены их ограничения в контексте безопасности и необходимости доверия между сторонами.
\item Создан прототип сервиса, для обмена активами между блокчейнами.
\item Проведено модульное и системное тестирование, подтверждающее работоспособность и безопасность разработанного сервиса.
\end{enumerate}

Разработанный сервис и результаты анализа могут быть использованы в финансовых организациях, работающих с международными расчётами и нуждающихся в альтернативных механизмах расчётов, либо в проектах, связанных с децентрализованными финансами (DeFi), для обеспечения более предсказуемых механизмов при проведении сделок.

Разработанный сервис показал высокую эффективность и безопасность в обмене активами между изолированными распределёнными реестрами. Это делает предложенный подход перспективным для широкого применения в различных сферах, связанных с блокчейн-технологиями и децентрализованными финансами.