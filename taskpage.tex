\begin{center}
\hyphenpenalty=10000 % disable sloppy
%----------------------------------------------------------------------------------------
%    HEADING SECTIONS
%----------------------------------------------------------------------------------------

% Name of your university/college
Министерство науки и высшего образования Российской Федерации\\
Санкт-Петербургский политехнический университет Петра Великого\\
Институт компьютерных наук и кибербезопасности\\
[1.2cm]

\begin{flushleft}
    \hspace{8.5cm}УТВЕРЖДАЮ\\
    \hspace{8.5cm}ДИРЕКТОР ВШ КТиИС\\
    \hspace{8.5cm}\underline{\hspace{3.5cm}} В.А. СУШНИКОВ\\
    \hspace{8.5cm}«\underline{\hspace{1.0cm}}»\underline{\hspace{3.5cm}} 2024 г.\\[0.8cm]
\end{flushleft}

ЗАДАНИЕ\\
на выполнение выпускной квалификационной работы магистра\\[1.0cm]

Студенту Мартынов Семён Андреевичу, группа 5140901/21501\hspace{1cm}\\[0.8cm]


\begin{enumerate}[label*=\arabic*.]
\item Тема работы: «Разработка сервиса для проведения обмена активами между изолированными распределёнными реестрами без установления доверия»
\item Срок сдачи студентом законченной работы: 22.05.2024
\item Исходные данные по работе:
	\begin{enumerate}[label*=\arabic*.]
	\item Материалы по работе распределённым реестрам, их видам и стандартам.
	\item Документация по работе виртуальной машины EVM в блокчейн сети Ethereum.
	\item Черновик предложения по использованию атомарных свапов в блокчейн сети Биткоин.
	\end{enumerate}
\item Содержание работы (перечень подлежащих разработке вопросов):
	\begin{enumerate}[label*=\arabic*.]
	\item Обзор существующих решений, включая исследование вопроса доверия между сторонами
	\item Формирование требований и ограничений к прототипу
	\item Описание бизнес-логики и архитектуры решения
	\item Программная реализация прототипа
	\item Модульное и системное тестирование
	\end{enumerate}
\item Консультанты по работе:\\ консультант по нормоконтролю – А.Г.Новопашенный
\item Дата выдачи задания 25.04.2024
\end{enumerate}

\vspace{0.5cm}

\begin{flushleft}
Руководитель ВКР \hspace{2.7cm} \underline{\hspace{3cm}} \hspace{0.4cm} к.т.н., доцент А.В. Самочадин

\vspace{0.5cm}

Задание принял к исполнению 25.04.20024\\
Студент \hspace{5.0cm} \underline{\hspace{3cm}} \hspace{0.4cm} С.А. Мартынов

\end{flushleft}
\end{center}
\newpage
